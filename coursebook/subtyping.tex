\newlecture

\begin{instructor}

  \section*{Lecture Outline}

  \begin{itemize}

  \item Show singleton pattern in Java and then in Scala.
  \item Show case objects, using \lstinline|Leaf| of a binary tree as an example.
  \item Introduce $A <: B$ notation.
  \item Intuition: the subtype can add more behavior, but cannot contradict or remove behavior.
  \item Example hierarchy: \lstinline|Animal| with \lstinline|sound| method. \lstinline|Dog extends Animal| and defines \lstinline|bite|.
    \lstinline|Cat extends Animal| and defines \lstinline|scratch|. \lstinline|Poodle extends Dog| and overrides \lstinline|sound|.
  \item Write out several subtyping relations and give examples of programs that do not typecheck.
  \item Introduce the \lstinline|Any| and \lstinline|Nothing| types and their relationship to all other types.
  \item Introduce a generic container with a single cell and get/set methods.
  \item Cannot stores cats in a \lstinline|C[Dog]|, since we may try to extract and \lstinline|.bite|.
  \item Cannot pass a \lstinline|C[Dog]| to a function that expects a \lstinline|C[Animal]|.
  \item We are going to do \emph{declarative-site variance}.
  \item Show that a Scala list of dogs can be treated as a list of animals.
  \item Introduce the co-variance annotation: \lstinline|+T|. This makes the setter untypable.
  \item Define the \lstinline|List| type in the right way with \lstinline|Nil| as a case object.
  \end{itemize}
  
\end{instructor}

\section{Reading}

Programming in Scala, Chapters 12 and 19.\footnote{\url{http://www.artima.com/pins1ed/}}

\section{The Singleton Pattern}

\begin{figure}
\begin{javacode}
public class Singleton {
  private static Singleton instance = null;

  private Singleton() { }

  public static Singleton getInstance() {
    if(instance == null) {
      instance = new ClassicSingleton();
    }
    return instance;
  }

  void myMethod() {
    System.out.println("It works")
  }

}
\end{javacode}
\caption{The Singleton Pattern in Java}
\label{javasingleton}
\end{figure}

At times, it is often desirable to only have a single instance of a particular
class. For example, if you have a class that lets you read input from the
user, it doesn't make sense to have two instances, since the user just has
one keyboard. You can create a singleton object in Java by following the
design pattern in \cref{javasingleton}. The only instance of \javainline{Singleton}
that can exist is the one stored in the private field. (Notice that the constructor
is private.) The only way to get
the singleton is by calling the \scalainline{Singleton.getInstance} method.

Here is the equivalent in Scala:

\begin{scalacode}
object Singleton {
  def myMethod() = println("It works")
}
\end{scalacode}

So, all the top-level functions we've created so far (in class and in
assignment) can be thought of as methods of singleton objects.

\section{Case Objects}

We've seen several examples of case classes that take no arguments. For example,
to represent binary trees with values at nodes (and not at the leaves), we could
write the following type:
%
\begin{scalacode}
sealed trait BinTree
case class BinTree(lhs: BinTree, value: Int, rhs: BinTree) extends BinTree
case class Leaf() extends BinTree
\end{scalacode}

However, we can also represent leaves using a \emph{case object}, which is essentially
a singleton that can be used in pattern-matching:
%
\begin{scalacode}
sealed trait BinTree
case class BinTree(lhs: BinTree, value: Int, rhs: BinTree) extends BinTree
case object Leaf extends BinTree
\end{scalacode}

Since \scalainline{Leaf} is an object and not a class, it \emph{cannot}
take any arguments, which why we can simply write \scalainline{Leaf} instead
of \scalainline{Leaf()}. Scala uses case objects extensively. For example,
\scalainline{Nil} and \scalainline{None} are case objects. If they
were case classes, we've have to write \scalainline{Nil()} and
\scalainline{None()} instead.

\section{Subtyping}

Subtyping in Scala is very similar to subtyping in Java.

\begin{notation}
We write $A <: B$ to mean $A$ is a subtype of $B$.
\end{notation}

The intuition behind subtyping is that a subtype always ``adds more features''
to its super-type, and doesn't ``subtract features''. Therefore, if $A <: B$,
then $A$ can be used in any context where $B$ is expected. The context simply
won't try to use the extra features of $A$. However, $B$ \emph{cannot}
be used in contexts where $A$ is expected, because the context may rely on
the ``added features'' of $A$ that $B$ does not provide.

For example, consider the following hierarchy of types, all of which
implement the \scalainline{Animal} trait\footnote{Think of a trait as an interface in Java. Traits
are actually more flexible, but we are only going to use them like interfaces for now.}
%
\begin{scalacode}
trait Animal {
  def sound(): String
}

class Dog extends Animal {
  def sound() = "woof"
  def bite() = "ouch"
}

class Poodle extends Dog {
  override def sound() = "yelp"
}

class Cat extends Animal {
  def sound() = "purr"
  def scratch() = "yow"
}
\end{scalacode}

A function that expects an \scalainline{Animal} can be applied to an object
of any class defined above. Similarly, a function that expects a \scalainline{Dog}
can be applied to a \scalainline{Dog} or a \scalainline{Poodle}, but not to a
\scalainline{Cat}. For example, the following function takes dogs and makes
them bite:
\begin{scalacode}
def dontBite(x: Dog) = {
  x.bite()
}
\end{scalacode}

The expression \scalainline{dontBite(new Cat())} will not type-check, which
is good, because cats don't have a \scalainline{bite} method.

Given the traits and types defined above, we can say that:
\begin{itemize}

  \item \scalainline{Dog <: Animal} because \scalainline{Dog extends Animal}

  \item \scalainline{Poodle <: Dog} because \scalainline{Poodle extends Dog}

  \item \scalainline{Cat <: Animal} because \scalainline{Cat extends Animal}

  \item \scalainline{Poodle <: Animal} because \scalainline{Poodle <: Dog} and
  \scalainline{Dog <: Animal} (subtyping is transitive),

  \item \scalainline{X <: X} for all $X$ (subtyping is reflexive).

\end{itemize}
%
In addition, scala has two special types:
%
\begin{itemize}

  \item \scalainline{X <: Any} for all types X.

  \item \scalainline{Nothing <: X} for all types X.

\end{itemize}

A peculiar property of \scalainline{Nothing} is that
\emph{there are no values with type} \scalainline{Nothing}!
It may seem pointless to have a type with no values. For example, the following
function cannot be applied to anything (not even to \scalainline{null}):
\begin{scalacode}
def useless(x: Nothing): Unit = {
  println("Cannot call me")
}
\end{scalacode}

But, we'll see why \scalainline{Nothing} is useful in a moment.

\section{Generics and Subtyping}

\begin{figure}
\begin{subfigure}[b]{0.26\textwidth}
\begin{scalacode}
def useDog(c: C[Dog]) = {
  c.get().bite()
}

useDog(new C(new Cat()))
\end{scalacode}
\caption{Cats can't bite.}
\label{catdog1}
\end{subfigure}
~\vrule~
\begin{subfigure}[b]{0.33\textwidth}
\begin{scalacode}
def useAnimal(c: C[Animal]) = {
  c.get().sound()
}

val c = new C[Dog](new Dog())
useAnimal(c)
\end{scalacode}
\label{catdog2}
\caption{Scala is being conservative.}
\end{subfigure}
~\vrule~
\begin{subfigure}[b]{0.31\textwidth}
\begin{scalacode}
def useAnimal2(c: C[Animal]) = {
  c.set(new Cat())
  c.get().sound()
}

val c = new C[Dog](new Dog())
useAnimal2(c)
c.get().bite()
\end{scalacode}
\label{catdog3}
\caption{Sneaky function confuses our code.}
\end{subfigure}
\caption{None of these programs type-check.}
\end{figure}

Subtyping becomes more complicated when we working with generics types, such
as lists, sets, or any other container type. To illustrate, we'll work
with the following generic type, which is the simplest possible container:
%
\begin{scalacode}
class C[T](private var x: T) {
  def get(): T = x
  def set(newX: T): Unit = x = newX
}

val c1: C[Dog] = new C(new Dog)
c1.set(new Poodle)
\end{scalacode}

Unsurprisingly, we can't store cats in \scalainline{c1}. If we did, a function
that consumes a container with a dog, might try to \scalainline{.get} the
dog and call the \scalainline{bite} method, which cats don't have.
Therefore, the code in \cref{catdog1} does not type-check.
However, since \scalainline{Dog <: Animal}, can we send a dog-container
to a function that expects an animal container. For example, \cref{catdog2}
appears to be safe. Unfortunately, this code does not type-check either. Although this particular example
is safe, consider the variation in \cref{catdog3}.

The problem above is that \scalainline{useAnimal2} sneakily stores a cat in
the container. After the function returns, \scalainline{c.get} would produce
a cat even though the type indicates that it should produce a dog.
Therefore, the Scala type-checker does not allow this program to type-check.

Unfortunately, \cref{catdog2} (which was safe) does not type-check either, just so that
this kind of unsafe example can be ruled out. Individual methods and functions
are type-checked only once. Similarly, when a function or method call is type-checked,
the body of the function is not re-examined.

\section{Variance}

\begin{instructor}
This is \emph{declaration-site variance}, as found in C\# and Scala. Java
does \emph{use-site variance}. John Altidor's papers summarize the differences
better than most stuff on the Web.
\end{instructor}

However, lists and other immutable data structures in Scala are not constrained
this way. For example, the following code does type-check:

\begin{scalacode}
def useAnimals(alist: List[Animal]) = {
  alist.map(animal => animal.sound()).mkString(", ")
}

val alist: List[Cat] = List(new Cat(), new Cat())
useAnimals(alist)
\end{scalacode}

The reason this works is because there are no methods on lists to update
their contents. The problem with our \scalainline{Container[T]} class is
that it writes to \scalainline{T}-typed values. However, if we had a functional
container class, we can use a \emph{covariance annotation} to indicate
that \scalainline{T}-typed values are never updated.

\begin{scalacode}
class Container[+T](private val x: T) {
  def get(): T = x
}
\end{scalacode}

The \scalainline{+T} annotation indicates \scalainline{Container[A] <:
Container[B]} when \scalainline{A <: B}. For this to be safe,
Scala ensures that \scalainline{T}-typed values are never updated
by the class.\footnote{Coavariance is actually more subtle than this. The reading discusses it in more depth, but you don't have to remember all the details.} (The class may have other kind of state, but it can't
update \scalainline{T}s.)

As a rule of thumb, almost any immutable data structure can be made
covariant, which makes it more flexible. E.g., we can a \scalainline{Set[Cat]}
where a \scalainline{Set[Animal]} is expected or a \scalainline{List[Dog]}
where a \scalainline{List[Animal]} is expected.

In fact, the list type in Scala is covariant:
%
\begin{scalacode}
sealed trait List[+A]
\end{scalacode}
Cons is easy to define as follows:
\begin{scalacode}
case class ::[A](head: A, tail: List[A]) extends List[A]
\end{scalacode}
However, the definition of \scalainline{Nil} is trickier. If it were a case-class,
we could write:
\begin{scalacode}
case class Nil[A]() extends List[A]
\end{scalacode}
However, recall that \scalainline{Nil} is a case-object. So, we may try to
write this:
\begin{scalacode}
case object Nil[A] extends List[A]
\end{scalacode}
However, objects can't have type parameters (or any parameters for that matter).

As discussed earlier, Scala has a special type \scalainline{Nothing} that has no values
and is the subtype of all other types: i.e.,
\scalainline{Nothing <: A}. Therefore, by covariance of lists, \scalainline{List[Nil] <: List[A]}, so
we can write:
\begin{scalacode}
case object Nil extends List[Nothing]
\end{scalacode}
Although there a no values of type \scalainline{Nothing}, an empty list of
of type \scalainline{List[A]} doesn't contain any values of type \scalainline{A}.
Therefore, the \scalainline{Nil} object can be given the type \scalainline{List[Nothing]}.

