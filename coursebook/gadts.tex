\newlecture

\begin{instructor}

\section*{Lecture Outline}

\begin{enumerate}

\item Write the Scala wrapper for Java's array class

\begin{scalacode}
case class AnyArray[T](elts: Array[T])
\end{scalacode}

Methods get, set, length, and foreach

\item Write bitarray. These are the get and set methods

\begin{scalacode}
    def get(index: Int): Boolean = elts(index >> 5) >> (index & 0x1F) == 1

    def set(index: Int, value: Boolean) = {

      if (value) {
        elts(index >> 5) = elts(index >> 5) | (1 << (index & 0x1F))
      }
      else {
        elts(index >> 5) = elts(index >> 5) & ~(1 << (index & 0x1F))
      }
\end{scalacode}

Lesson: any code that uses BitArray needs to change. e.g., we can't write a generic foreach method

\item Introduce the ArrayLike trait:

\begin{scalacode}
  trait ArrayLike[T] extends Any {
    def get(index: Int): T
    def set(index: Int, value: T): Unit
    def length(): Int
  }
\end{scalacode}

\item Natural to write:

\begin{scalacode}
case class BitArray[T](elts: Array[Int]) ArrayLike[T]
\end{scalacode}

But, we can instead write:

\begin{scalacode}
case class BitArray(elts: Array[Int]) ArrayLike[Boolean]
\end{scalacode}

\item We've specialized the type parameter, but that's ok. (It's like what we did
  for \scalainline{List[Nothing]}.)

\item Further optimization: we are allocating an object that wraps the underlying
  Java array. Fix: use AnyVal:

\begin{scalacode}
  case class AnyArray[T](elts: Array[T]) extends AnyVal with CompactArray[T]
  case class BitArray(elts: Array[Int]) extends AnyVal with CompactArray[Boolean]
\end{scalacode}

Basically, Scala can ``see through'' the wrapper type.

Lessons: low-level optimizations are necessary for very high-performance code. Exposing low-level abstractions makes code un-reusable. GADTs can help.

\end{enumerate}

\end{instructor}

\begin{figure}
\scalafile{code/Gadts.scala}
\caption{Using generics to abstract complex representations.}
\end{figure}
