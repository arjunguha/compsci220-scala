\newlecture

\textbf{Course Description}: Development of individual skills necessary for
designing, implementing, testing and modifying larger programs, including: use
of integrated design environments, design strategies and patterns, testing,
working with large code bases and libraries, code refactoring, and use of
debuggers and tools for version control.

\section{OK, but Why?}

Good style is very important. It makes code more uniform and readable, and thus
easier to debug by others, as well as avoids programming patterns that are error
prone.
\par
You may not always agree with every point in this style guide or any other style
guide; however, the ability to write code to match an arbitrary style guide is
an important skill that you are expected to have in the real world (and, yes, even
in research settings).

\section{Formatting}

\begin{itemize}
		\item \textbf{Whitespace}\\
		  Indent with two spaces. Lines should be no longer than 100 characters. Use one
      blank line between method, class, and object definitions. Please, please,
      please follow this, it makes code far easier to debug for everyone.
		\item \textbf{Naming}
		\begin{itemize}
			\item \textit{Use short names for small scopes}\\
			  It's all but expected that single character variable names will be used
        inside of simple lambda functions. Feel free to use longer variable names
        if you'd prefer, but it is not required.
			\item \textit{Use longer names for larger scopes}\\
			  Use descriptive names when providing methods such as external APIs. Don't write
        a sentence, but your method name ought not be a combination of random
        characters corresponding to the first letter of words either (e.g. the C standard
        library).
			\par
			In addition, please don't just increment function names. Naming \texttt{myFunction}
      and \texttt{myFunction2} obfuscates both the differences between functions as well
      as hides the actual behavior of the functions.
			\item \textit{Use descriptive names for methods that return values}\\
			Use names like \texttt{src.isDefined} not \texttt{src.defined}.		
		\end{itemize}
		\item \textbf{Braces}\\
		  Unlike Java, where you ought to always use curly braces for every statement, you
      can and should drop the braces for simple expressions.\\\textit{Example:} \\
    \scalafile{code/style_guide_good_braces.scala}
		not
    \scalafile{code/style_guide_bad_braces.scala}
		However, braces should continue to be used for more complex expressions.
	\end{itemize}

\section{Types and Generics}

Types are amazing. They allow you to explicitly define what data types should be in
what variables and allow you to catch many, many bugs at compile time. That said,
explicitly specifying types can also be redundant, especially in cases where types
are explicitly clear.
\par
As a result, types should be used judiciously; find the right balance between using
them everywhere and using them nowhere. If in doubt, include them.
\par
Being clever with types can lead to hard-to-debug issues. Scala is invariant by
default, and avoid using covariance unless explicitly asked to use it in an
assignment.

\section{Standard Library}

Don't use standard functions that you don't fully understand. We would much rather
see you re-implement a function that exists than see you chaining together several
standard functions in which you don't understand how they operate.
\par
That said, unless explicitly told otherwise, we would like you to try and use
the standard library. There are many standard library functions that students do end
up re-implementing. While this serves as a great learning experience, we also
\textit{highly} encourage you to go back and refactor your code to utilize standard
library functions. It will be educational and lead to terser code in the future.

\section{Control Flow}

Scala isn't Java; however, regardless of language, large blocks of control flow
are hard to grok and far more likely to be sources of logic bugs.
\par
If you have large chunks of logic, you ought to reconsider your approach. For the
most part, none of the methods we expect you to write ought to be more than four
or five lines of code. If you find yourself with 20+ lines, you ought to reconsider
your approach.
	\begin{itemize}
		\item \textit{Recursion}\\
		  Phrasing your code in terms of recursion often makes it terser. In addition,
      tail recursion in Scala often makes this approach just as performant as
      writing the code iteratively.
		\item \textit{Loops}\\
		  You can't use 'em and you don't need them. Simple as that. Recursion should
      replace the need to iteratively access elements in lists.
	\end{itemize}

\section{Function Declarations}
Don't declare functions inside other functions. It will lead to scoping bugs that
are very difficult to debug.
	
\section{Error Handling}
Unlike CS187, and much like real life, we do not explicitly tell you how to handle corner
cases. Write code in such a way to minimize corner cases, and then handle the cases
in a sane manner. We are not going to try to nail you on corner cases in your code,
but don't do something crazy.
		
\section{Test Driven Development}
Write tests often, they will help with problems. If you see erroneous behavior,
write a test case to replicate it. We will tell you throughout this course if
you come to us with broken code, you should write unit tests to debug it.
