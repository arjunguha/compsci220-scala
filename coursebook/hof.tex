\newlecture

\section{Generics and Type Inference\classtime{15}}

\begin{figure}
\begin{minipage}{0.45\textwidth}
\begin{scalacode}
def ilength(alist : List[Int]): Int = alist match {
  case Nil => 0
  case _ :: tail => 1 + ilength(tail)
}
\end{scalacode}
\subcaption{Length of a list of integers.}
\end{minipage}
\quad\vrule\quad
\begin{minipage}{0.45\textwidth}
\begin{scalacode}
def slength(alist : List[String]): Int = alist match {
  case Nil => 0
  case _ :: tail => 1 + slength(tail)
}
\end{scalacode}
\subcaption{Length of a list of integers.}
\end{minipage}
\caption{Two very similar length functions.}
\label{monolength}
\end{figure}

\Cref{monolength} shows two functions that calculate (1) the length of a list
of integers and (2) the length of a list of strings. These two functions
are almost identical. The only difference between them in the type of
element in the list. We can abstract away the differences by writing
a \emph{generic} length function:
%
\begin{scalacode}
def length[A](alist: List[A]): Int = alist match {
  case Nil => 0
  case _ :: tail => 1 + length(tail)
}
\end{scalacode}

In the code above, \verb|A| is a \emph{type argument}. It truly is an argument
to the function, just like \verb|alist|. i.e., when \scalainline{length} is
invoked, we need to specify \verb|A|:
%
\begin{scalacode}
length[Int](List(1, 2, 3))
length[String](List("A", "B", "C", "D"))
\end{scalacode}
%
However, \verb|A| holds a type instead of a value.

Scala lets you elide type arguments in most cases. So, we could also write:
%
\begin{scalacode}
length(List(1, 2, 3))
length(List("A", "B", "C", "D"))
\end{scalacode}
%
But, it is important to note that under the hood, Scala figures out that we
meant \scalainline{A} to be \scalainline{Int} on the first line and
\scalainline{String} on the second line and inserts the type arguments for us.
In fact, we relied on this in the definition of \scalainline{length}: Scala
expands the recursive call \scalainline{length(tail)} to
\scalainline{length[A](tail)}. Note that Scala isn't just ``guessing'' the
right type. It uses the type of the value-arguments to infer the missing
type argument.

However, Scala's type inference is not perfect. It saves you typing, but don't
expect it to figure out exactly what you mean. It can make mistakes.
For example, if we write \scalainline{List(1, "2", 3)}, Scala infers
that it has type \scalainline{List[Any]}. But, this is almost certainly
not what we meant! We probably meant to construct a list of integers
or strings. If we instead wrote \scalainline{List[Int](1, "2", 3)}, Scala
would have signalled a type error.

\subsection{Generic Data Structures}

\begin{figure}

\begin{minipage}{0.45\textwidth}
\begin{scalacode}
sealed trait IBinTree
case class ILeaf() extends IBinTree
case class INode(
  lhs: IBinTree,
  key: Int,
  value: Int,
  rhs: IBinTree)
  extends IBinTree
\end{scalacode}
\caption{Values are integers.}
\end{minipage}
\quad\vrule\quad
\begin{minipage}{0.45\textwidth}
\begin{scalacode}
sealed trait SBinTree
case class SLeaf() extends IBinTree
case class SNode(
  lhs: SBinTree,
  key: Int,
  value: String,
  rhs: SBinTree)
  extends SBinTree
\end{scalacode}
\caption{Values are strings.}
\end{minipage}
\caption{Two non-generic binary trees.}
\label{monobintrees}
\end{figure}

\Cref{monobintrees}
In the last lecture, we build a data structure that represents binary
search trees with strings for values. We can write a similar data structure
with integer-values, as shown in \cref{monobintrees}. But, this will force
us to redefine the functions we wrote earlier to proceed \scalainline{SBinTree}s.
Instead, we can write a generic type for binary trees where the values have
some unknown type \scalainline{A}:

%
\begin{scalacode}
sealed trait BinTree[A]
case class Leaf[A]() extends BinTree[A]
case class Node[A](lhs: BinTree[A], key: Int, value: A, rhs: BinTree[A]) extends BinTree[A]
\end{scalacode}

Here are some examples of binary trees with values of different types:

\begin{itemize}

\item This binary tree has integer values:
\begin{scalacode}
Node(Leaf(), 10, 500, Leaf())
\end{scalacode}

\item This binary tree has string values:
\begin{scalacode}
Node(Leaf(), 10, "five hundred", Leaf())
\end{scalacode}

\item This binary tree has boolean values:
\begin{scalacode}
Node(Leaf(), 10, true, Leaf())
\end{scalacode}

\end{itemize}

We can easily update the \scalainline{insert}, \scalainline{find}, and \scalainline{size}
functions from the last lecture to work with generic binary trees.

\section{Higher-Order Functions on Lists}


\begin{figure}
\begin{subfigure}[b]{.45\textwidth}
\begin{scalacode}
def incrs(alist: List[Int]): List[Int] =
  alist match {
    case Nil => Nil
    case h :: t => h + 1 :: incrs(t)
  }
\end{scalacode}
\caption{A function that increments all numbers in a list.}
\end{subfigure}
\quad\vrule\quad
\begin{subfigure}[b]{.45\textwidth}
\begin{scalacode}
def lengths(alist: List[String]): List[Int] =
  alist match {
  case Nil => Nil
  case h :: t => h.length :: lengths(t)
}
\end{scalacode}
\caption{A function that calculates the lengths of all strings in a list.}
\end{subfigure}
\quad
\vskip 1em \hrule \vskip 1em
\begin{subfigure}[b]{.45\textwidth}
\begin{scalacode}
def doubles(alist: List[Int]): List[Int] =
  alist match {
    case Nil => Nil
    case h :: t => h * 2 :: doubles(t)
}
\end{scalacode}
\caption{A function that doubles all numbers in a list.}
\end{subfigure}
\quad\vrule\quad
\begin{subfigure}[b]{.45\textwidth}
\begin{scalacode}
def negates(alist: List[Int]): List[Int] =
  alist match {
    case Nil => Nil
    case h :: t => -h :: negates(t)
}
\end{scalacode}
\caption{A function that negates all numbers in a list.}
\end{subfigure}

\caption{Four different functions that transform elements of a list.}
\label{map_motivation}
\end{figure}


\subsection{The Map Function\classtime{15}}

Study the four functions in \cref{map_motivation}. The function \verb|incrs|
adds one to every element in a list of integers, the function \verb|lengths|
calculates the length of every string in a list, the function \verb|doubles|
doubles every number in a list, and the function \verb|negates| negates every
number in a list. Hopefully, you've noticed that these four functions have a lot
of in common. \emph{The only difference between them is the operation that they
perform on the head of the list}.

Here is a variant of \verb|doubles| that makes the operation explicit, by
moving it into a separate function:
%
\begin{scalacode}
def f(n: Int): Int = n * 2

def doubles(alist: List[Int]): List[Int] = alist match {
  case Nil => Nil
  case h :: t => f(h) :: doubles(t)
}
\end{scalacode}

In this version, we're simply applying a function $f$ to $h$, where $f$
is the doubling function. \emph{You should apply the same refactoring to the other
functions}. E.g.,  instead of directly writing \scalainline{head.length}, refactor
the function so that this expression is in a helper function.

Once we've re-written the operation as \scalainline{f(h)}, all three
functions look identical: the only difference is that each refers to a
different function.
Instead of writing three functions that are almost identical, we can
write one function that takes $f$ as an argument:

\begin{scalacode}
def map(f: Int => Int, alist: List[Int]): List[Int] = alist match {
  case Nil => Nil
  case h :: t => f(h) :: map(f, t)
}
\end{scalacode}

With this function, we can make our examples much more succinct:

\begin{scalacode}
def doubles(alist: List[Int]) = {
  def f(n: Int): Int = n * 2
  map(f, alist)
}

def incrs(alist: List[Int]) = {
  def f(n: Int): Int = n + 1
  map(f, alist)
}

def lengths(alist: List[String]) = {
  def f(str: String): Int = str.length
  map(f, alist)
}
\end{scalacode}

Unfortunately, the definition of \scalainline{lengths} above does not
type-check. Scala reports two type errors and they are both very informative:

\begin{console}
<console>:15: error: type mismatch;
 found   : String => Int
 required: Int => Int
         map(f, alist)
             ^
<console>:15: error: type mismatch;
 found   : List[String]
 required: List[Int]
         map(f, alist)
                  ^
\end{console}

The \scalainline{map} function only works of lists of integers. However, if you
look at the definition of \scalainline{map} closely, you'll see that all the
\scalainline{Int}-specific code has been factored out into \scalainline{f}. We
can make the function even more generic by introducing two type-parameters:

\begin{scalacode}
def map[A,B](f: A => B, alist: List[A]): List[B] = alist match {
  case Nil => Nil
  case h :: t => f(h) :: map[A, B](f, t)
}
\end{scalacode}

The type-parameter $A$ is the type of the elements in
\emph{alist} and the type-parameter $B$ is the type elements in the produced list.
In the code above, we've make all the type-arguments explicit. But, as we've
discussed before, we can rely on type-inference to fill them in:

\begin{scalacode}
def map[A,B](f: A => B, alist: List[A]): List[B] = alist match {
  case Nil => Nil
  case h :: t => f(h) :: map(f, t)
}
\end{scalacode}

\subsection{The Filter Function\classtime{10}}

Another common pattern when programming with lists is to select certain elements
that have some property. For example, here is a function that consumes a list,
and produces a new list that only contains the even numbers:

\begin{scalacode}
def filterEven(alist: List[Int]): List[Int] = alist match {
  case Nil => Nil
  case head :: tail =>
    (head % 2 == 0) match {
      case true => head :: filterEven(tail)
      case false => filterEven(tail)
    }
}
\end{scalacode}

This is a very common pattern too. For example, we could write a function to
select the odd numbers or the prime numbers. If we had a list of strings, we
could select all the strings with length 5 or all the strings that represent
English-language words. All these functions have the same shape: they test
the value of \scalainline{head} in some way. If the test succeeds, \scalainline{head} is added in
the output list. But, if the test fails, it is excluded.

Following the same strategy we used to derive \scalainline{map}, we first package
the \scalainline{head}-test into a function:

\begin{scalacode}
def f(n: Int): Boolean = head % 2

def filterEven(alist: List[Int]): List[Int] = alist match {
  case Nil => Nil
  case head :: tail =>
    f(head) match {
      case true => head :: filterEven(tail)
      case false => filterEven(tail)
    }
}
\end{scalacode}

Now that the pattern is clearer, we generalize \scalainline{filterEven} to take \scalainline{f}
as an argument:

\begin{scalacode}
def filter(f: Int => Boolean, alist: List[Int]): List[Int] = alist match {
  case Nil => Nil
  case head :: tail =>
    f(head) match {
      case true => head :: filter(f, tail)
      case false => filter(f, tail)
    }
}

def filterEven(alist: List[Int]): List[Int] = {
  def f(n: Int): Boolean = head % 2
  filter(f, alist)
}
\end{scalacode}

Finally, just as we did for \scalainline{map}, we can generalize the type of \scalainline{filter}
so that it can be applied to \scalainline{List[A]}:

\begin{scalacode}
def filter[A](f: A => Boolean, alist: List[A]): List[A] = alist match {
  case Nil => Nil
  case head :: tail =>
    f(head) match {
      case true => head :: filter(f, tail)
      case false => filter(f, tail)
    }
}
\end{scalacode}

\begin{instructor}
Cannot show find because it relies on Option
\end{instructor}

% \subsection{The Find Function\classtime{10}}

% Another common operation on lists is to find an element. For example,
% the following function finds the first even number in a list:

% \begin{scalacode}
% def findEven(alist: List[Int]): Option[Int] = alist match {
%   case Empty() => None()
%   case Cons(head, tail) => (head % 2 == 0) match {
%     case true => Some(head)
%     case false => findEven(tail)
%   }
% }
% \end{scalacode}

% If we change \scalainline{(head % 2 == 0)} to \scalainline{(head % 2 == 1)}, we'll have the
% \scalainline{findOdd} function. Again, instead of defining two slightly different
% functions, we can write the general \scalainline{find} function:

% \begin{scalacode}
% def find[A](f: A => Boolean, alist: List[A]): Option[A] = alist match {
%   case Empty() => None()
%   case Cons(head, tail) => f(head) match {
%     case true => Some(head)
%     case false => find(f, tail)
%   }
% }
% \end{scalacode}

\subsection{Sorting}

You've written sorting functions that sort lists is ascending or descending
order. But, the following higher-order function can be used to sort any
lists of any type, given an arbitrary ``less than'' operator.

\begin{scalacode}
def insert[A](lessThan: (A, A) => Boolean, x: A, alist: List[A]): List[A] = alist match {
  case Nil => List(x)
  case hd :: tl => {
    if (lessThan(x, hd)) {
      x :: hd :: tl
    }
    else {
      hd :: insert(lessThan, x tl)
    }
  }
}

def sort[A](lessThan: (A, A) => Boolean, alist: List[A]) = alist match {
  case Nil => Nil
  case hd :: tl => insert(lessThan, hd, sort(lessThan, tl))
}

def sortAscending(alist: List[Int]): Int = {
  def f(x: Int, y: Int): Boolean = x < y
  sort(f, alist)
}

def sortDescending(alist: List[Int]): Int = {
  def f(x: Int, y: Int): Boolean = x > y
  sort(f, alist)
}

def sortByLengthAscending(alist: List[String]): Int = {
  def f(x: String, y: String): Boolean = x.length < y.length
  sort(f, alist)
}
\end{scalacode}

\section{Builtin Methods on Lists\classtime{5}}

All the higher-order list-processing functions that we've defined are already
built-in to Scala as methods on lists.

\begin{itemize}

\item
We can map over a list using the \scalainline{map} method:

\begin{scalacode}
def f(n: Int): Int = n + 1

assert(List(10, 20, 30).map(f) == List(11, 21, 22))
\end{scalacode}

\item We can filter a list using the \scalainline{filter} method:

\begin{scalacode}
def f(n: Int): Boolean = n % 2 == 0

assert(List(0, 1, 2, 3, 4, 5).map(f) == List(0, 2, 4))
\end{scalacode}

\item We can sort a list using the \scalainline{sortWith} method:

\begin{scalacode}
def f(x: Int, y: Int): Boolean = x > Y

assert(List(3, 7, 9, 11).sortWith(f) == List(11, 9, 7, 3))
\end{scalacode}

\end{itemize}

There are several other higher-order functions that you can discover from
the documentation or the Scala console.
