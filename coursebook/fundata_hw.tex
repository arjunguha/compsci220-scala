\newhw{Functional Data Structures}

In this assignment, you'll build two data structures that use functional
programming ideas in a unique way.
\begin{instructor}
Say something more motivating!
\end{instructor}

\section{Preliminaries}

You should create a directory-tree that looks like this:

\dirtree{%
.1 ./fundata.
.2 project.
.3 plugins.sbt.
.2 src.
.3 main.
.4 scala.\DTcomment{Your solution goes here}.
.3 test.
.4 scala\DTcomment{Yours tests go here}.
}

The \texttt{project/plugins.sbt} file must have exactly this line:

\scalafile{../hw/fundata/template/project/plugins.sbt}

You should implement your solution in a single file, within an object called
\scalainline{FunctionalDataStructures}.

\begin{figure}
\scalafile{../hw/fundata/template/src/main/scala/Solution.scala}
\caption{Solution template.}
\end{figure}

\section{Persistent Queues}

Recall from earlier classes, that a \emph{queue} is a data structure that
supports three operations:

\begin{enumerate}

\item \emph{Empty} constructs an empty queue,

\item \emph{Enqueue} adds a new element to the back of the queue,

\item \emph{Dequeue} removes an element from the front of the queue, if the queue is
  not empty.

\end{enumerate}

In the following exercises, you will build a \emph{persistent queue}. A
persistent queue has the operations defined. But, instead of having enqueue and
dequeue update the queue, they leave the original queue unchanged and return a
new queue.

It is easy to implement a persistent queue using a list:

\begin{scalacode}
type SlowQueue[A] = List[A]

def emptySlow[A](): SlowQueue[A] = Nil()

def enqueueSlow[A](elt: A, q: SlowQueue[A]): SlowQueue[A] = q match {
  case Nil => List(elt)
  case head :: tail => head :: enqueueSlow(elt, tail)
}

def dequeueSlow[A](q: SlowQueue[A]): Option[(A, SlowQueue[A])] = q match {
  case Nil => None()
  case head :: tail => Some((head, tail))
}
\end{scalacode}

Read the code above carefully. The \emph{enqueue} operation traverses the
entire list each time (i.e., $O(n)$ running time). Your task is to implement
the queue more efficiently.

The trick is to represent the queue using two lists. The first list, called
\emph{front}, has the elements at the front of the queue. The second list, called
\emph{back}, has the elements at the back of the queue, \emph{in reverse order}.

For example, if \emph{front} is \scalainline{List(1, 2, 3)} and \emph{back} is \scalainline{List(6, 5, 4)}, then
the elements of the queue, in order, are 1, 2, 3, 4, 5, 6. With this
representation:

\begin{itemize}

\item \emph{Enqueue} adds an element to \emph{back}, but doesn't need to
  traverse the whole list.

\item \emph{Dequeue} removes an element from \emph{front}, unless \emph{front}
  is empty. If it is empty, it reverses \emph{back} and uses it as the front.

\end{itemize}

In your solution, use the following type to represent queues:

\begin{scalacode}
case class Queue[A](front: List[A], back: List[A])
\end{scalacode}

Then, implement the following functions:

\begin{scalacode}
def enqueue[A](elt: A, q: Queue[A]): Queue[A]

def dequeue[A](q: Queue[A]): Option[(A, Queue[A])]
\end{scalacode}

\section{Join Lists}

A \emph{join list} is a data structure that represents a list, but the
elements are arranged into a tree, as follows:

\begin{scalacode}
sealed trait JoinList[A] { val size: Int }
case class Empty[A]() extends JoinList[A] { val size = 0 }
case class Singleton[A](elt: A) extends JoinList[A] { val size = 1 }
case class Join[A](lst1: JoinList[A], lst2: JoinList[A], size: Int) extends JoinList[A]
\end{scalacode}

This tree-shape makes some operations,
like list-concatenation very efficient.
The type has three constructors:

\begin{enumerate}

\item \scalainline{Empty()} represents an empty list.

\item \scalainline{Singleton(x)} represents a list with one element $x$.

\item \scalainline{Join(lst1, lst2, length)} represents \scalainline{lst1} appended to
       \scalainline{lst2}. The \scalainline{length} field is the total number of
       elements in the list.

\end{enumerate}

It should be clear that it is very cheap to append two join lists: you simply
use the \scalainline{Join} constructor. It is also cheap to calculate the length of a
join list, since it is stored at each node.


Finally, since join lists represent lists, it is easy to convert between
join lists and normal lists.
%
\begin{itemize}

\item \scalainline{toList[A](lst: JoinList[A]): List[A]} converts a join list into a Scala
  list.  This
  operation can be very expensive, but is useful for testing.

\item \scalainline{fromList[A](lst: List[A]): JoinList[A]} converts a Scala list into a join
  list by repeatedly splitting a list into two equal halves.

\end{itemize}

We've provided functions to do so in \cref{joinlist_tests}.
You may use these functions in your test suites. You must not use them in your
solution.

\begin{figure}
\scalafile{../hw/fundata/template/src/test/scala/Tests.scala}
\caption{Helpers for writing tests.}
\label{joinlist_tests}
\end{figure}

\paragraph{Programming Task}

Your task is to write some typically list-processing functions for join lists.

\begin{enumerate}

\item \scalainline{max(lst, compare)} produces the maximum value in
  \scalainline{lst}. The second argument is a comparator. If
  \scalainline{compare(x, y) == true}, then \scalainline{x} is greater than
  \scalainline{y}. If the list is empty, the function produces
   \scalainline{None}.

\item \scalainline{map(f, lst)} produces a new join list, which has exactly the same
 shape as \scalainline{lst}, but with \scalainline{f} applied to every element.

\item \scalainline{filter(pred, lst)} produces a new join list that has only
  includes elements of \scalainline{lst} that satisfy the given predicate.
  The order of elements should not change.

\item \scalainline{first(lst)} and \scalainline{rest(lst)} produce the head and tail, respectively,
   of \scalainline{lst} if it is non-empty.

\item \scalainline{nth(lst, i)} produces the $n$th element of the list (the first element has index 0).

\end{enumerate}

\section{Hand In}

From \sbt{}, run the command \texttt{submit}. The command will create
a file called \texttt{submission.tar.gz} in your assignment directory.
Submit this file using Moodle.

For example, if the command runs successfully, you will see output similar
to this:
%
\begin{console}
Created submission.tar.gz. Upload this file to Moodle.
[success] Total time: 0 s, completed Jan 17, 2016 12:55:55 PM
\end{console}

\textbf{Note:}  The command will not allow you to submit code that does not
compile. If your code doesn't compile, you will receive no credit for the
assignment.


