\newlecture

\textbf{Course Description}: Development of individual skills necessary for
designing, implementing, testing and modifying larger programs, including: use
of integrated design environments, design strategies and patterns, testing,
working with large code bases and libraries, code refactoring, and use of
debuggers and tools for version control.

\section{Introduction\classtime{15}}

\emph{CMPSCI220 Programming Methodology} introduces you to all the concepts
above in the context of a modern programming language:
\href{http://www.scala-lang.org/what-is-scala.html}{Scala}.
You could use Scala to write exactly the same kind
of object-oriented code that you've seen in Java. In fact, Scala code and Java
code can seamlessly co-exist and interoperate in the same program; we'll
leverage this feature later in the course. In fact, many of the design patterns
that you will learn in this course will be applicable to Java and Scala.

However, a key reason we're using Scala is to expose you to programming techniques
and language features that are beyond the scope of Java. Most
modern software systems are written in a plethora of languages. In fact, large
systems tend to use several programming languages. Therefore, to succeed in your
computing career, you have to be familiar with several languages and be able to
learn new languages on your own. Programming
languages are constantly invented and abandoned\footnote{\href{http://www.oreillynet.com/pub/a/oreilly/news/languageposter_0504.html}{This poster} is a very incomplete history of the birth and death of programming languages.}
 and it is impossible to predict the next big language that everyone
will use or the language you'll need to learn for your first job.

Scala is a big language with many unique features and we are not going to learn
to use them all. Instead, we are going to focus on ideas that Scala shares with
other modern programming languages. Here are some of the key ideas that we will
cover in this course that go beyond Java:
%
\begin{itemize}

\item \emph{First-class functions} are the cornerstone of \emph{functional programming}.
  They are pervasive in JavaScript, Ruby, Python, Swift, and almost all modern
  languages.

  In fact, even \href{{http://docs.oracle.com/javase/tutorial/java/javaOO/lambdaexpressions.html}}{Java} and
  \href{http://msdn.microsoft.com/en-us/library/dd293608.aspx}{C++} recently
  added support for first-class functions.

\item \emph{Algebraic data types} are available in Apple Swift, Mozilla Rust,
  Microsoft F\#, and several other programming languages. Programming with
  algebraic data types is very different from programming in an object-oriented
  style. We'll cover both styles of programming in this course and develop
  a deep understanding of the tradeoffs.

\item \emph{Type inference} is available in modern typed programming languages,
  such as C\# and Swift, and even in a limited form in
  \href{http://docs.oracle.com/javase/tutorial/java/generics/genTypeInference.html}{Java}.
  As the name suggests, in a language with type inference, the compiler can
  often ``infer'' or fill-in types that you omit. So, your programs become shorter, but
  retain all the advantages of type checking.

\end{itemize}

The main themes of the course are not language-specific. We will emphasize
the following broad ideas that are applicable to all software development:
%
\begin{itemize}

\item \emph{Testing} is critical for building reliable software. You will learn how
  to test complex functions and make effective use of testing tools and frameworks.
  Every programming problem you solve in this course will have to be tested. We
  expect you to write good tests yourself. The quality of your tests will be a
  significant portion of your grade on every assignment.

\item \emph{Design patterns} are recipes for solving typical programming
  problems. This course will emphasize object-oriented and functional
  design patterns. We will focus on design patterns that are applicable to
  a variety of programming languages, and not Scala-specific design patterns.

\item \emph{Refactoring} is a key concept that we emphasize throughout the
  course. As we introduce new ideas, we will systematically refactor our old
  code to exploit them.

\item \emph{Debugging} is a necessary skill because even small programs often
  have bugs.

\item \emph{Command-line tools} such as compilers and build tools lie under the
  hood of sophisticated IDEs such as Eclipse. Learning to use the command-line
  will make you a better IDE user. Moreover, since many new languages lack good
  IDEs, we will emphasize the use of command-line tools in this course.

\item \emph{Version control} software is critical for collaborative software
  development and used by all professional programmers. Although you will be
  programming alone in the course, version control will still help you organize
  your programming and save you a lot of time if you accidentally delete or
  break your code.

\item \emph{Using libraries} is critical for writing software that gets real
  work done. Initially, you'll use libraries that were developed specifically
  for this course, but you will eventually learn to discover and use
  libraries from the Web.

\end{itemize}

The overarching goal of this course is to make you a better programmer,
and an important part of that is to get familiar with programming terminology
and culture. Unfortunately, there is a lot of misinformation on the Web
about programming, but we will try to point you to sources that are reliable.
Here are two good places to start:
%
\begin{itemize}

  \item \href{http://paulgraham.com/articles.html}{Paul Graham's Essays}.
  The earlier essays are particularly pertinent, E.g.,
  \href{http://paulgraham.com/avg.html}{Beating the Averages} and
  \href{http://paulgraham.com/popular.html}{Being Popular}.

  \item \href{http://www.joelonsoftware.com}{Joel Spolsky's blog}. E.g.,
  \href{http://www.joelonsoftware.com/articles/CollegeAdvice.html}
  {Advice for Computer Science College Students}
  and \href{http://www.joelonsoftware.com/articles/ResumeRead.html}
  {Getting Your Resume Read}.

\end{itemize}

Finally, \href{http://xkcd.com}{XKCD} comics often make obscure programming
references and this course will help you decipher some of them.

\section{The Command-Line\classtime{5}}

The \emph{Linux command-line}\footnote{The command-line is also known as a
\emph{terminal} or \emph{shell}.} is  a critical part of this course. If you're
using the course virtual-machine, you should use the program \textbf{LXTerminal}
to start the command-line.

Unless you're already familiar with the Linux command-line, you must read
\href{http://learncodethehardway.org/cli/book/cli-crash-course.html}{Zed Shaw's Command Line Crash Course},
up to and including the chapter ``Removing a File (rm)''. Zed likes to swear at
his own readers, so we'd like to apologize in advance on his behalf. The rest
of these lecture notes assume that you are familiar with the command-line.

\begin{instructor}
\begin{itemize}

\item Introduce the course VM.

\item  Introduce the command line (\verb|mkdir|, \verb|cd|, and, \verb|ls|).
\end{itemize}

\end{instructor}

\section{\sbt{} and the Scala REPL\classtime{5}}

\sbt{} is the Swiss Army Knife of Scala programming. It is a command-line tool
that can be use to run Scala programs, compile them, test them, package them
deployment, publish them to the Web, and more. Like many modern programming
languages, \sbt{} has a \emph{REPL} (read-eval-print loop), which you can use to
type in and run one-line programs immediately, without the bother of creating
files, etc.

To start the Scala REPL, open a terminal (LXTerminal on the course VM),
type in \verb|sbt console| and press enter. Your screen will look like this:
%
\consolefile{includes/sbtconsole.txt}

The \verb|scala>| prompt indicates that you can type in Scala expressions to evaluate.

\section{Scala Basics\classtime{15}}

\subsection{Simple Expressions and Names}

Arithmetic in Scala is very similar to arithmetic in Java:

\begin{console}
scala> 19 * 17
res0: Int = 323
\end{console}

Strings in Scala will also look familiar:

\begin{console}
scala> "Hello, " + "world"
res1: String = Hello, world
\end{console}

Boolean expressions will be familiar too:

\begin{console}
scala> true && false
res2: Boolean = false
\end{console}

Let's examine the last interaction more closely. When you type \scalainline{true && false},
Scala prints three things:
%
\begin{itemize}
\item An automatically-generated \emph{name} (\scalainline{res2}),
\item The \emph{type} of the expression (\scalainline{Boolean}), and
\item The \emph{value} of the expression (\scalainline{false}).
\end{itemize}

On the Scala REPL, you can use the generated name as a variable. But, you're
better off picking meaningful names yourself using \scalainline{val}:

\begin{console}
scala> val mersenne = 524287
mersenne : Int = 524287
scala> val courseName = "Programming Methodology"
courseName : String = Programming Methodology
\end{console}

\subsection{Type Inference}

You'll find that Scala programs are significantly shorter than their Java
counterparts. A key feature of Scala that lets you write less code is \emph{type
inference}. Notice in the variable definitions above, you did not have to write
any types. Instead, Scala \emph{inferred} them for you. This feature is very helpful
in larger programs, where types can become complex.

Type inference is not magic; later in the course, you'll learn more about how it
works and when it doesn't. For now, here's a rule of thumb: Scala can infer the
type of variable named with \verb|val|. But, Scala \emph{cannot} infer the type of
function parameters.

\subsection{Functions}

Here is a very simple Scala function:

\begin{console}
scala> def double(n: Int): Int = n + n
double: (n: Int)Int
\end{console}

This code defines a function called \scalainline{double}, which takes an argument called
\scalainline{n} of type \scalainline{Int} and returns a value of type \scalainline{Int}. We can apply the
function as follows:

\begin{console}
scala> double(10)
res3: Int = 20
\end{console}

The following function takes two arguments, \scalainline{x} and \scalainline{y} and calculates the
distance from the point \scalainline{(x,y)} to the origin:

\begin{console}
scala> def dist(x: Double, y: Double): Double = math.sqrt(x * x + y * y)
dist: (x: Double, y: Double)Double

scala> dist(3.0, 4.0)
res4: Double = 5.0
\end{console}

Notice that unlike variable definitions, we need \emph{type annotations}
on function parameters and result types.

If your function actually fits on a line (without scrolling off your window),
you can define them very tersely as shown above. But, many interesting
functions span several lines and need local variables.

\subsection{Blocks and Local Variables}

You can define local variables within a \emph{block}. A block is code delimited by
curly-braces. For example:

\begin{console}
scala> def dist2(x: Double, y: Double): Double = {
  val xSq = x * x
  val ySq = y * y
  math.sqrt(xSq + ySq)
}
\end{console}

\begin{figure}
\scalafile{code/Lecture1fac.scala}

\caption{A Scala module}
\label{lecture1code}
\end{figure}

\section{\sbt{} Project Structure\classtime{5}}

In principle, you can write a full-fleged program line-by-line in the Scala
console. But, it makes a lot more sense to save large programs to files
for a particular project. To do so, we will being by creating a new directory
for your project.

First, exit the Scala console by typing \verb|:quit| and then exit \sbt{}
by typing \verb|exit|. You should return to the command-line:
%
\begin{console}
scala> :quit

[success] Total time: 3 s, completed Jan 13, 2016 8:47:38 PM

sbt> exit
student@vm:~$
\end{console}

At the terminal, let's create a directory for the project:
%
\begin{console}
student@vm:~$ mkdir lecture1
\end{console}
%
Then, let's enter the directory:
\begin{console}
student@vm:~$ cd lecture1
student@vm:~/lecture1$
\end{console}
Notice that the name of the directory is displayed on the command-line.

Your Scala projects will have two kinds of files: test cases and your implementation.
\sbt{} requires you to organize your files into the following directory
structure:
\dirtree{%
.1 ~/lecture1.
.2 src.
.3 main.
.4 scala\DTcomment{Implementation goes here}.
.3 main.
.4 scala\DTcomment{Tests goes here}.
}

You can create these directories by running the following commands:
\consolefile{includes/mkdirs.txt}

We will now see how to save Scala code to files. Using a text editor
(e.g., Sublime Text), create a file called \verb|Lecture1.scala| in the
\verb|src/main/scala| directory, with the contents shown in \cref{lecture1code}.
The code creates an object with a single function to calculate factorials.

\begin{instructor}
Scala wart: Do not \verb|extend App| because it can't be evaluated in console.
\end{instructor}

When you write functions in a Scala file, you \emph{have} to place it in
an object. You cannot just write \scalainline{def fac ...} without an enclosing
object. This is a peculiarity of Scala that we will explain later in the course.
In this example, the name of the object is ``Lecture1'', but it can be anything
you like.

Now that we've saved this function to a file, we can use it from the console:
%
\begin{console}
sbt> console
scala> import Lecture1._
scala> fac(10)
\end{console}

\begin{figure}
\begin{scalacode}
// src/main/scala/Lecture1Tests.scala
import Lecture1._

class Lecture1Tests extends org.scalatest.FunSuite {

  test("fac -- base case") {
    assert(fac(0) == 1)
  }

  test("fac -- inductive case") {
    assert(fac(5) == 120)
  }

}
\end{scalacode}
\caption{Unit tests for the code in \cref{lecture1code}}
\label{lecture1tests}
\end{figure}

\section{Testing\classtime{5}}

The \sbt{} console is a convenient way to experiment with new code or write
a ``one off' functionn. However, you must write \emph{unit tests} to test
any actual code you write. \Cref{lecture1tests} shows an example of
unit tests that use the \emph{ScalaTest} library. The code is quite self-explanatory:
each test suite is a class that extends \scalainline{org.scalatest.FunSuite}.
The body of the class has several test blocks, as shown in the figure.

\begin{instructor}

\begin{itemize}

\item Propose a new function to write. Write the test-cases first, then
the implementation and re-run \verb|sbt test|.

\item Recap the development methodology: first write tests cases, then
write the implementation, and re-test continuously. Show continuous
testing with \verb|~test|.

\end{itemize}


\end{instructor}

\section{Building and Pattern-Matching on Lists\classtime{25}}


In this section, we will show how to write simple list-processing functions. We
will cover basic \emph{functional programming} and introduce \emph{pattern
matching}.

\subsection{Constructing Lists}

The simplest list is the empty list, which is written in Scala as
%
\begin{scalacode}
Nil
\end{scalacode}
%
Given the empty list, we can construct larger lists using the \scalainline{::}
operator (which is pronounced \emph{cons}). Here is a simple example that
constructs a one-element list:
%
\begin{scalacode}
50 :: Nil
\end{scalacode}
Given a one-element list, we can build a two-element list by using the
\scalainline{::} operator again:
%
\begin{scalacode}
100 :: (50 :: Nil)
\end{scalacode}
%
We can use \scalainline{::} again to build a three-element list:
%
\begin{scalacode}
200 :: (100 :: (50 :: Nil))
\end{scalacode}
%
In an expression \scalainline{x :: y},
\scalainline{x} is known as the \emph{head} of the list and \scalainline{y} is
known as the \emph{tail}. Note that the tail of a list is always a list itself,
(though it may be the empty list \scalainline{Nil}).

For example, consider the list below:
\begin{scalacode}
val letters = "a" :: ("b" :: :: Nil)
\end{scalacode}
%
\begin{itemize}

  \item The head of \scalainline{letters} is \scalainline{"a"}.

  \item The tail of \scalainline{letters} is \scalainline{"b" :: Nil}.

  \item The head of the tail of \scalainline{letters} is \scalainline{"b"}.

  \item The tail of the tail of \scalainline{letters} is \scalainline{Nil}.

  \item \scalainline{Nil} does not have a head or a tail.

\end{itemize}

In our examples so far, we've used parenthesis to make the head and tail clear.
However, you can simply write \scalainline{"a" :: "b" :: "c" :: Nil}. Intuitively,
everything to the right of a \scalainline{::} is the tail. If you get confused
up, write the parenthesis explicitly.

It is usually easier to write lists in the following way:
%
\begin{itemize}

  \item \scalainline{List("a", "b", "c")} is equivalent to
  \scalainline{"a" :: ("b" :: ("c" :: Nil))}.

  \item \scalainline{List()} is equivalent to \scalainline{Nil}.

\end{itemize}
%
However, it is important to understand that this is just a convenient notation.
Under the hood, Scala transforms these expressions to use \scalainline{::}
and \scalainline{Nil}, as we described above.

\paragraph{Lists and Type Inference}

You should try to type out the expressions above into the \sbt{} console.
For example:
%
\begin{console}
scala> val lst = 1 :: 2 :: 3 :: Nil
lst: List[Int] = List(1, 2, 3)
\end{console}
%
As you can see, Scala prints lists using the shorthand notation,
even if you explicitly use \scalainline{::} and \scalainline{Nil}.
More significantly, Scala has inferred that the type of the list
is \scalainline{List[Int]}. There was no need to explicitly state that
is is the case.

Here is another example, where Scala infers that the type of a list is
\scalainline{List[String]}:
%
\begin{console}
scala> val lst = List("a", "b", "c")
lst: List[String] = List("a", "b", "c")
\end{console}

Type inference is very convenient and spares you from having to explicitly
specify the type of the element. However, type inference is not magic
and can behave in unexpected ways. For example, in the interaction
below, Scala infers that the type of the list is \scalainline{List[Any]}:
%
\begin{console}
scala> val lst = List("a", 10, "c")
lst: List[Any] = List("a", 10, "c")
\end{console}
Although this is technically true, if you write this code, it is more
likely that you made a mistake and intended to actually create a list of
strings. If you're ever unsure, you can write the type explicitly, which
would signal a type error in this case:
\begin{console}
scala> val lst = List[String]("a", 10, "c")
<console>:10: error: type mismatch;
 found   : Int(10)
 required: String
       val lst = List[String]("a", 10, "c")
                                   ^
\end{console}

\begin{figure}

\begin{subfigure}[b]{0.45\textwidth}
\begin{scalacode}
def countDown(n: Int): List[Int] = {
  if (n == 0) {
    Nil
  }
  else {
    n :: countDown(n - 1)
  }
}
\end{scalacode}
\caption{}
\label{countDown}
\end{subfigure}
\hskip 2em
\begin{subfigure}[b]{0.45\textwidth}
\begin{scalacode}
def fromTo(lo: Int, hi: Int): List[Int] = {
  if (lo == hi) {
    lo :: Nil
  }
  else {
    lo :: fromTo(lo + 1, hi)
  }
}
\end{scalacode}
\caption{}
\label{fromTo}
\end{subfigure}

\caption{Functions that produce lists.}
\label{listcons}
\end{figure}

\paragraph{Functions that produce lists}

Now that we've seen how to construct lists explicitly, it is straightforward
to write functions that produce lists. \Cref{listcons} shows some simple
recursive functions that construct new lists.

\begin{figure}
\begin{subfigure}[b]{0.45\textwidth}
\begin{scalacode}
def product(lst: List[Int]): Int = lst match {
  case Nil => 1
  case n :: rest => n * product(rest)
}
\end{scalacode}
\caption{Calculate the product of a list of numbers.}
\end{subfigure}
\hskip 2em
\begin{subfigure}[b]{0.45\textwidth}
\begin{scalacode}
def repeatTwice(lst: List[Int]): List[Int] = lst match {
  case Nil => Nil
  case n :: rest => n :: n :: repeatTwice(rest)
}
\end{scalacode}
\caption{Repeats every element of a list twice.}
\end{subfigure}

\caption{Two simple functions that consume lists.}
\label{listconsumers}
\end{figure}


\subsection{Pattern Matching}

Now that we've seen how to write functions that produce lists, we'll
learn how to write functions that consume lists as arguments. We'll
start by writing a simple function to calculate the sum of a list of numbers.
Here are some examples of of \scalainline{sum} being used:
%
\begin{scalacode}
assert(sum(20 :: 30 :: Nil) == 50)
assert(sum(1 :: 2 :: 3 :: Nil) == 6)
\end{scalacode}

Intuitively, to calculate \scalainline{sum(1 :: 2 :: 3 :: Nil)}, we can
can recursively calculate the sum of the tail and add that value to the
head:
\begin{scalacode}
   sum(1 :: 2 :: 3 :: Nil)
== 1 + sum(2 :: 3 :: Nil)
== 1 + (2 + sum(3 :: Nil))
== 1 + (2 + (3 + sum(Nil)))
\end{scalacode}
%
The last line shows an important special case. Since the empty list
doesn't have a head or a tail, we need to treat it differently. We'll say
that \scalainline{sum(Nil)} is \scalainline{0}.

We can write \scalainline{sum} by using a powerful feature of
Scala called \emph{pattern matching}.
%
\begin{scalacode}
def sum(lst: List[Int]): Int = {
  lst match {
    case Nil => 0
    case h :: t => h + sum(t)
  }
}
\end{scalacode}
%
This code makes it clear that the function is inspecting \scalainline{lst}
and considering two \emph{cases}. When \scalainline{lst} is \scalainline{Nil},
it produces \scalainline{0} and when \lstinline{lst} is constructed with
the \scalainline{::} operator, the function recurs on the tail and ands
that result to the head. In this code,
\scalainline{Nil} and \scalainline{h :: t} are called \emph{patterns}.

\Cref{listconsumers} shows two more list-consuming functions that
use pattern-matching. Notice that in \scalainline{::} cases, these functions
use different variable names for the head and the tail. A pattern can use
any variable to refer to the value of the head or the tail.

\begin{figure}
\begin{subfigure}[b]{0.45\textwidth}
\begin{scalacode}
def countOnes(lst: List[Int]): Int = lst match {
  case Nil => 0
  case n :: rest => {
    if (n == 1) {
      1 + countOnes(rest)
    }
    else {
      countOnes(rest)
    }
  }
}
\end{scalacode}
\caption{Counting ones using an if-expression.}
\label{countonescomplex}
\end{subfigure}
\hskip 2em
\begin{subfigure}[b]{0.45\textwidth}
\begin{scalacode}
def countOnes(lst: List[Int]): Int = lst match {
  case Nil => 0
  case 1 :: rest => 1 + countOnes(rest)
  case n :: rest => countOnes(rest)

}
\end{scalacode}
\caption{Counting ones using a composite pattern.}
\label{countonessimple}
\end{subfigure}

\caption{Pattern matching can make complex conditionals clearer.}
\end{figure}

\paragraph{Complex Patterns}
%
Pattern-matching is extremely powerful and can be used to express
complex conditionals. For example, \cref{countonescomplex} is a function
that counts the number of \scalainline{1}s that occur in a list: it uses
pattern matching as introduced above and then an if-expression to
check if the head of the list is \scalainline{1}.

\Cref{countonessimple}
is the same function, rewritten to use pattern matching. This version is shorter
and makes it clear that there are three cases. To do so, we exploit the fact
that patterns can match \emph{almost} any value, including numbers,
strings, lists, and user-defined data structures too (which we will see
in a later class).

\paragraph{Exhaustivity and Reachability Checking}
%
Here is another function that uses pattern matching to count the number
of tens in a list:
%
\begin{scalacode}
def countTens(lst: List[Int]): Int = lst match {
  case 10 :: rest => 1 + countTens(rest)
  case n :: rest => countTens(rest)
}
\end{scalacode}

However, this function had a bug. Do you see it? If you type this into
a console, Scala prints the following:
%
\begin{console}
<console>:10: warning: match may not be exhaustive.
It would fail on the following input: Nil
       def countTens(lst: List[Int]): Int = lst match {
                                            ^
\end{console}
%
Scala has detected that we forgot to write a case for \scalainline{Nil}.
Scala ensures that your patterns are \emph{exhaustive}. Here is another
buggy version of the function:
%
\begin{scalacode}
def countTens(lst: List[Int]): Int = lst match {
  case n :: rest => 1 + countTens(rest)
  case n :: rest => countTens(rest)
  case Nil => 0
}
\end{scalacode}
%
In this version, we wrote the patterns incorrectly, so the first and second
patterns are identical. Scala reports the following error:
\begin{scalacode}
<console>:13: warning: unreachable code
         case n :: rest => countTens(rest)
\end{scalacode}
%
Scala ensures that all cases are \emph{reachable}.

This automatic exhaustivity and reachability checking makes programs that
use pattern-matching much more robust than programs that use complicated,
nested if-statements. Pattern matching is a very powerful tool that you can
exploit to make your programs more robust. We will emphasize pattern-matching
over conditionals in this course.


