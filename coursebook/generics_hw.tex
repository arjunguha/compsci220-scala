\newhw{Generics}

This assignment will exercise your knowledge of generic interfaces,covariance
and bounded quantification.

\section{Preliminaries}

You should create a directory-tree that looks like this:

\dirtree{%
.1 ./generics.
.2 project.
.3 plugins.sbt.
.2 src.
.3 main.
.4 scala.
.5 Solution.scala.\DTcomment{Your solution goes here}.
.3 test.
.4 scala\DTcomment{Yours tests go here}.
}

The \texttt{project/plugins.sbt} file must have exactly this line:

\begin{scalacode}
addSbtPlugin("edu.umass.cs" % "compsci220" % "1.0.2")
\end{scalacode}

The support code for this assignment is in the package \texttt{hw.generics}.

\section{Programming with Bounded Quantification}

The trait \scalainline{hw.generics.ListLike} is a trait for ``list-like''
collections. (i.e., collections that are either \emph{empty} or have
a  \emph{head} and \emph{tail}.) The class \scalainline{hw.generics.MyList}
is a typical list that implements the \scalainline{ListLike} trait.

\begin{enumerate}

  \item In \texttt{Solution.scala}, create the following type for
  binary trees:
  \begin{scalacode}
  sealed trait BinTree[A]
  case class Node[A](lhs: BinTree[A], value: A, rhs: BinTree[A]) extends BinTree[A]
  case class Leaf[A]() extends BinTree[A]
  \end{scalacode}

  Furthermore, have \scalainline{BinTree} extend \scalainline{ListLike}.
  The \emph{head} of a binary-tree is the value on the left-most node
  and the \emph{tail} of a binary-tree is the tree with the left-most
  value removed.

  \item In \texttt{Solution.scala}, create an object called \scalainline{ListFunctions} with the following functions:

    \begin{enumerate}

      \item Create a function \scalainline{filter(f, alist)}
      where \scalainline{alist} is a list-like collection and
      \scalainline{f} is a predicate that can be applied to elements in
      the list. The function should produce a new list-like collection
      with the same type as \scalainline{alist} that only contains the
      elements on which \scalainline{f} produces \scalainline{true}.

      i.e., this filtering function should behave exactly the same
      as Scala's filtering function.

      \item Create a function \scalainline{append(alist1, alist2)}, where
      \scalainline{alist1} and \scalainline{alist2} are two list-like
      collections of the same type. The result should be a new list-like
      collection (with the same type as \scalainline{alist1} and
      \scalainline{alist2}) that has the elements of \scalainline{alist1}
      followed by the elements of \scalainline{alist2} in order.

      i.e., on linked lists, \scalainline{append(alist1, alist2)} should
      behave in a manner simlar to \scalainline{alist1 ++ alist2}.

      \item Define a function that sorts in ascending order with the
      following name and type:

      \begin{scalacode}
      def sort[A <: hw.generics.Ordered[A], C <: hw.generics.ListLike[A, C]](alist: C): C = {
        ...
      }
      \end{scalacode}

    \end{enumerate}

    You should test applying these functions to the provided
    \scalainline{MyList} type and to the \scalainline{BinTree} type that
    you defined. You should also be able to apply it any other new data
    structure that extends the \scalainline{ListLike} trait.

\end{enumerate}

\section{Extending Traits}

The package \scalainline{hw.generics} defines three traits \scalainline{T1},
\scalainline{T2}, and \scalainline{T3}
that define exactly the same
set of methods, but have a different set of type-parameters.
In \texttt{Solution.scala}, create the following classes:

\begin{scalacode}
class C1 {
  def f(a: Int, b: Int): Int = 0
  def g(c: String): String = ""
  def h(d: String): Int = 0
}


class C2 {
  def f(a: Int, b: Int): Int = 0
  def g(c: Int):  Int = 0
  def h(d: Int): Int = 0
}

class C3[A](x: A) {
  def f(a: Int, b: A): Int = 0
  def g(c: A): String = ""
  def h(d: String): A = x
}

class C4[A](x: Int, y: C4[A]) {
  def f(a: Int, b: C4[A]): C4[A] = b
  def g(c: Int): C4[A] = y
  def h(d: C4[A]): Int = x
}
\end{scalacode}

Moreover, have these classes extend as many of the traits
traits \scalainline{T1}, \scalainline{T2}, and \scalainline{T3}
as possible. Some classes may be able to extend several
traits. \textbf{You should not modify the class bodies.}

\section{Hand In}

From \sbt{}, run the command \texttt{submit}. The command will create
a file called \texttt{submission.tar.gz} in your assignment directory.
Submit this file using Moodle.

For example, if the command runs successfully, you will see output similar
to this:
%
\begin{console}
Created submission.tar.gz. Upload this file to Moodle.
[success] Total time: 0 s, completed Jan 17, 2016 12:55:55 PM
\end{console}

\textbf{Note:}  The command will not allow you to submit code that does not
compile. If your code doesn't compile, you will receive no credit for the
assignment.



\section{Template}

You can use the following template for \texttt{Solution.scala}:
\begin{scalacode}
import hw.generics._

sealed trait BinTree[A]
case class Node[A](lhs: BinTree[A], value: A, rhs: BinTree[A]) extends BinTree[A]
case class Leaf[A]() extends BinTree[A]

object ListFunctions {
  // def filter(f, alist)
  // def append(alist1, alist2)
  def sort[A <: Ordered[A], C <: ListLike[A, C]](alist: C): C = ???
}

class C1 {
  // Do not change the class body. Simply extend T1, T2, and/or T3.
  def f(a: Int, b: Int): Int = 0
  def g(c: String): String = ""
  def h(d: String): Int = 0
}

class C2 {
  // Do not change the class body. Simply extend T1, T2, and/or T3.
  def f(a: Int, b: Int): Int = 0
  def g(c: Int):  Int = 0
  def h(d: Int): Int = 0
}


class C3[A](x: A) {
  // Do not change the class body. Simply extend T1, T2, and/or T3.
  def f(a: Int, b: A): Int = 0
  def g(c: A): String = ""
  def h(d: String): A = x
}

class C4[A](x: Int, y: C4[A]) {
  // Do not change the class body. Simply extend T1, T2, and/or T3.
  def f(a: Int, b: C4[A]): C4[A] = b
  def g(c: Int): C4[A] = y
  def h(d: C4[A]): Int = x
}
\end{scalacode}
