\newlecture

\begin{instructor}

\section*{Lecture Outline}
\classtime{15}

\begin{enumerate}

  \item Recap:
\begin{itemize}
  
  \item The notation \scalainline{A <: B}.
  \item \emph{Invariance}: typically \scalainline{A <: B} does not imply that \scalainline{Container[A] <: Container[B]}, because the container may have a method that writes to a value of type A.
  \item \emph{Covariance}: If the class is annotated \scalainline{Container[+T]}, then the type-checker ensures we can't write to values of type \scalainline{T} within the class.
  \item Traits can take type parameters too, just like classes. When we extend a trait, we need to supply the type parameters.
  \item Bounded quantification: The type parameter\scalainline{C[A]} means that \scalainline{A} could be any type. We can write \scalainline{C[A <: B]} to restrict it to types that extend \scalainline{B}, which allows us to invoke \scalainline{B}-methods on values of type \scalainline{A}.
\end{itemize}

\item Define hierarchy with \lstinline|Dog| and \lstinline|Poodle|. Poodles should have a \lstinline|bite| method
  that dogs do not.

\item Define container of two objects \lstinline|class Two[+T](x: T, y: T)|, thus \lstinline|Two[A] <: Two[B]| if \lstinline|A <: B|.

\item Add functional update operations, \emph{but do not compile}.

\item This code fragment would go wrong:

\begin{lstlisting}
def storeDog[A <: Two[Dog]](x: A): A = x.update1(new Dog)
val twoPoodles = new Two[Poodle](new Poodle(), new Poodle())
val aPoodle = storeDog(twoPoodles).get1() // Produces a Dog, not a Poodle!
aPoodle.bite() // Crash! Dogs don't have a .bite() method.
\end{lstlisting}

\item The problem is that when a \lstinline|Two[A]| is treated as a
  \lstinline|Two[B]| (where A <: B), we can update the container to
  store a B, which breaks code that expects A-typed values.

\item Solution:

  \begin{lstlisting}
    def update1[S >: T](newX: T): Two[S] = new Two(newX, y)
  \end{lstlisting}

  Reason through the types when updating with a Dog or a Poodle.

\item The same problem exists with list-append.

\item A model of type-checking:

  \begin{itemize}

  \item Define AST for numbers, booleans, and let-binding (called EVal). Include pretty-printer.
  \item Define types: ints and bools
  \item Define a type-checker without an environment.
  \item Add the environment.     

  \end{itemize}

\end{enumerate}
\end{instructor}

\section{Variance and method arguments\classtime{20}}

Earlier, we wrote the simplest container we could imagine that only stored
a single value. The following container slightly more sophisticated, because
it can store two values:
\begin{scalacode}
class Two[+T](private val x: T, private val y: T) {
  def get1(): T = x
  def get2(): T = y
}
\end{scalacode}
This container uses a covariance annotation, so when \scalainline{A <: B},
we have \scalainline{Two[A] <: Two[B]}. Recall that the
intuition for subtyping is that \scalainline{Two[A]} can be used
wherever a \scalainline{Two[B]} is expected. Now, suppose we modify the class to
add functional update methods:
\begin{scalacode}
class Two[+T](private val x: T, private val y: T) {
  ...
  def update1(newX: T): Two[T] = new Two(newX, y)
  def update2(newY: T): Two[T] = new Two(x, newY)
}
\end{scalacode}
This is a natural operation on containers and we should be able to write it.
But, let's ensure that even with this operation, \scalainline{Two[A]} can
be used whenever a \scalainline{Two[B]} is expected.

Concretely, consider the following two classes:
\begin{scalacode}
class Dog {
  def makeSound(): String = "woof"
}

class Poodle extends Dog {
  def bite(): String = "nip"
}
\end{scalacode}

We should be able to use a \scalainline{Two[Poodle]} in all contexts where
a \scalainline{Two[Dog]} is expected.


Consider the following function, which simply updates a container to
store a \scalainline{Dog}:
\begin{scalacode}
def storeDog[A <: Two[Dog]](x: A): A = x.update1(new Dog)
\end{scalacode}
However, it uses bounded-quantification to any subtype of
\scalainline{Two[Dog]}. Therefore, we can apply it to a \scalainline{Poodle}
container and know that the result is still a \scalainline{Poodle}
container. Of course, this isn't true, since the function stores
a \scalainline{Dog}. Therefore, the following code goes wrong:
\begin{scalacode}
val twoPoodles = new Two[Poodle](new Poodle(), new Poodle())
val aPoodle = storeDog(twoPoodles).get1() // Produces a Dog, not a Poodle!
aPoodle.bite() // Crash! Dogs don't have a .bite() method.
\end{scalacode}

Naturally, Scala (and Java) won't allow this to happen. In general, when
a class has a covariant type parameter \scalainline{T}, its methods can
produce values of type \scalainline{T}, but methods' arguments cannot consume
values of type \scalainline{T} (because of the error described above). Therefore, the type-checker
will not allow us to write the update methods, because they use
a covariant type parameter as the type of a method argument.

\paragraph{A solution}
The problem with the code above is that when a \scalainline{Two[A]}
is treated as a \scalainline{Two[B]}, we can update the container to
store any \scalainline{B}-typed value, which then breaks code that expects
\scalainline{A}-typed values. Instead, we need to ensure that we can only store
subtypes of \scalainline{A} in the container. We can express this constraint
using bounded-quantification:

\begin{scalacode}
class Two[+T](private val x: T, private val y: T) {
  ...
  def update1[S >: T](newX: T): Two[S] = new Two(newX, y)
  def update2[S >: T](newY: T): Two[S] = new Two(x, newY)
}
\end{scalacode}
This version of the class does type-check.

If we have a Poodle-container:
\begin{scalacode}
val x = new Two[Poodle](new Poodle, new Poodle)
\end{scalacode}
And we update one of the poodles:
\begin{scalacode}
x.update1[Poodle](new Poodle)
\end{scalacode}
The type-parameter \scalainline{S} is bound to the type \scalainline{Poodle},
so the return type is \scalainline{Two[Poodle]}. However, we can also
update the the container with a \scalainline{Dog}:
\begin{scalacode}
x.update1[Dog](new Dog)
\end{scalacode}
However, in this case, the return type is \scalainline{Two[Dog]} (although one
of the dogs is a poodle).

\paragraph{Appending Lists}
A similar problem arises with list concatenation. Since lists are covariant,
they cannot have an append method with this type:
\begin{scalacode}
sealed trait List[A] {
  def append[A](other: List[A]): List[A] = ...
  ...
}
\end{scalacode}

However, we can use the same trick we used above to write an append
method with the following type:

\begin{scalacode}
sealed trait List[A] {
  def append[B :> A](other: List[B]): List[B] = alist match {
    case Nil => Nil
    case hd :: tl => hd :: tl.append(other)
  }
  ...
}
\end{scalacode}

\section{A Model of Type-Checking\classtime{40}}

\begin{figure}
\scalafile{includes/ScalaFragment.scala}
\caption{A subset of Scala expressions}\label{scalafragment}
\end{figure}

\begin{instructor}
This section is quite rough. I'd start by getting students to derive the expression data-structure from examples.
\end{instructor}


To understand generics, we've had to develop an intuition for how the Scala
type-checker works. A deep understanding of the Scala (or Java) type system is
beyond the scope of this class. However, it is important to have a reasonably
accurate mental model of type-checking to understand and debug type-errors.

\Cref{scalafragment} is a data structure that represents a fragment of
Scala expressions, including numbers, boolean, addition, the less-than operator,
if-expressions, and identifiers (bound with \scalainline{val}). Since
the only values in this fragment are integers and booleans, we can represent
types as follows:

\begin{scalacode}
sealed trait Type
case object TInt extends Type
case object TBool extends Type
\end{scalacode}

Ignoring identifiers, we can write a simple recursive function to type-check
programs in this fragment, as shown in \cref{simpletc}.

\begin{figure}
\begin{scalacode}
object TypeError extends RuntimeException("Type error")

object TypeChecker {

  def tc(expr: Expr): Type = expr match {
    case EInt(_) => TInt
    case EBool(_) => TBool
    case EAdd(e1, e2) => (tc(e1), tc(e2)) match {
      case (TInt, TInt) => TInt
      case _ => throw TypeError
    }
    case ELT(e1, e2) => (tc(e1), tc(e2)) match {
      case (TInt, TInt) => TBool
      case _ => throw TypeError
    }
    case EIf(e1, e2, e3) => (tc(e1), tc(e2), tc(e3)) match {
      case (TBool, t1, t2) if (t1 == t2) => t1
      case _ => throw TypeError
    }
  }
}
\end{scalacode}
\caption{Type-checking, excluding identifiers}\label{simpletc}
\end{figure}

Type-checking identifiers is a little trickier. When we see an identifier,
there is way to determine its type, unless we remember the type
of the expression it was bound to. Therefore, we need an auxiliary
parameter, known as the environment, to ``remember'' the type of identifiers
so that we can recall them later (\cref{tcid}).

\begin{figure}
\begin{scalacode}
object TypeChecker {
  def tc(env: Map[String, Type], expr: Expr): Type = expr match {
    case EInt(_) => TInt
    case EBool(_) => TBool
    case EAdd(e1, e2) => (tc(env, e1), tc(env, e2)) match {
      case (TInt, TInt) => TInt
      case _ => throw TypeError
    }
    case ELT(e1, e2) => (tc(env, e1), tc(env, e2)) match {
      case (TInt, TInt) => TBool
      case _ => throw TypeError
    }
    case EIf(e1, e2, e3) => (tc(env, e1), tc(env, e2), tc(env, e3)) match {
      case (TBool, t1, t2) if (t1 == t2) => t1
      case _ => throw TypeError
    }
    case EVal(x, e1, e2) => tc(env + (x -> tc(env, e1)), e2)
    case EId(x) => env.getOrElse(x, throw TypeError)
  }
}
\end{scalacode}
\caption{Type-checking identifiers.}\label{tcid}
\end{figure}

The actual Scala type-checker is vastly more complicated. But, it follows
this basic design.
