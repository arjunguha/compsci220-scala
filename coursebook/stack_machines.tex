\newlecture

\begin{instructor}

\section*{Lecture Outline}

\begin{enumerate}

  \item Introduction: Over the next two lectures, we'll look under the
hood and discuss some of the ideas used to build the Scala compiler
and the Java virtual machine.  Most of the time, you will not have to
think about these kinds of low-level details.  However, to write
performance-critical code or to debug low-level errors, you will need
to know this kind of work.

  \item Ambiguity in parsing arithmetic expressions

  \item In Scala \lstinline|1 + 2| and \lstinline|1.+(2)| mean the same thing

  \item Concrete vs. abstract syntax: draw the abstract syntax trees for arithmetic expressions

  \item Define the \lstinline|Expr| datatype. (Include subtraction!)

  \item A \emph{parser} takes a concrete syntax string and turns it into an abstract syntax tree

  \item Scala AST example:

\begin{lstlisting}
import scala.tools.reflect._
import runtime.universe._
val tb = runtimeMirror(getClass.getClassLoader).mkToolBox()
val tree = tb.parse("""(x: Int, y: Int, z: Int) => x + y * z""")
tb.eval(tree)
res0.asInstanceOf[(Int, Int, Int) => Int](1, 2, 3)
\end{lstlisting}

  \item Write recursive evaluator \lstinline|def eval(e: Expr): Int|, but this is not how Scala is evaluated

  \item Show Java compilation process:
   and show that \lstinline|javac| produces the .class file
  and that \lstinline|java|. TalEclipse, etc. do this under the hood.

  \item 

  \begin{enumerate}

    \item Write a class with static main  and factorial method

    \item Show that \texttt{javac} produces the class file

    \item Show that \texttt{java} runs the class file and it works even if the source code is deleted

    \item An IDE uses this stuff under the hood

    \item Disassemble the classfile using \texttt{javap -c Simple.class} and show the instruction stack

    \item Show the class files that Scala generates

  \end{enumerate}

  \item Define instructions and write
  \lstinline|eval(code: List[Instruction], stack: List[Int]): List[Int]|
  and \lstinline|compile(e: Expr): List[Instruction]|
  


\end{enumerate}

\end{instructor}

Over the next two lectures, we'll look under the hood and
discuss some of the ideas used to build the Scala compiler and the Java virtual machine.
Most of the time, you will not have to think about these kinds of low-level details.
However, to write performance-critical code or to debug low-level
errors, you will need to know this kind of work.


\section{Programs as Trees}

These are examples of arithmetic expressions:

\begin{itemize}

\item \verb|2 * 4|

\item \verb|1 + 2 + 3|

\item \verb|5 * 4 * 2|

\item \verb|1 + 2 * 3|

\end{itemize}
%

We all know how to evaluate these expressions in our heads. But, when we do,
we resolve several ambiguities. For example, should we evaluate \verb|1 + 2 * 3|
like this:

\begin{tabular}{ll}
& \verb|1 + 2 * 3| \\
$=$ & \verb|(1 + 2) * 3| \\
$=$ & \verb|3 * 3| \\
$=$ & \verb|9|
\end{tabular}

\noindent or like this:

\begin{tabular}{ll}
& \verb|1 + 2 * 3| \\
$=$ & \verb|1 + (2 * 3)| \\
$=$ & \verb|1 + 6| \\
$=$ & \verb|7|
\end{tabular}

The latter is the convention in mathematics, but it is an arbitrary choice.
A programming language could adopt either convention or adopt a completely
different notation. For example, the following three programs, written
in three different languages, are trying to express the same thing:
%
\begin{itemize}

\item \verb|1 + 2| infix syntax, in Scala, which (mostly) adopts the notation of
   conventional mathematics

\item \verb|(+ 1 2)| parenthesized prefix syntax, from the Scheme language

\item \verb|1 2 +| postfix syntax, from the Forth language

\end{itemize}
%
These three \emph{concrete syntaxes} are very different, but all mean
``the sum of the number three and the number four''.

Concrete syntax is important, because it is the human-computer interface to a
programming language. It is easy to find acrimonious debates on the Web about
the virtues of Python's indentation-sensitive syntax, whether semicolons should
be optional in JavaScript, how C code should be indented, and so on. However,
we can think of expressions more abstractly as \emph{abstract syntax trees}.  

Scala's case classes make it easy to define a type for abstract syntax trees
of arithmetic expressions:
%
\begin{scalacode}
sealed trait Expr
case class Num(n: Int) extends Expr
case class Add(e1: Expr, e2: Expr) extends Expr
case class Sub(e1: Expr, e2: Expr) extends Expr
case class Mul(e1: Expr, e2: Expr) extends Expr
\end{scalacode}

Here are some examples of abstract arithmetic expressions and their
concrete representations (written in normal, mathematical notation):

\begin{tabular}{l|l}
Concrete Syntax & Abstract Syntax \\
\hline
\texttt{1 + 2 + 3} & \lstinline[language=scala]|Add (Add (Num(1), Num(2)), Num(3))| \\
\texttt{1 + 2 * 3} & \lstinline[language=scala]|Add (Num(1), Mul (Num(2), Num(3)))| \\
\texttt{(1 + 2) / 3} & \lstinline[language=scala]|Div (Add (Num(1), Num(2)), Num(3))|
\end{tabular}

The Scala compiler uses a much larger and more complex type to represent Scala
abstract syntax trees, but it follows the same principle described here. You can
use some low-level Scala APIs to take a string and turn it into an abstract syntax
tree and even evaluate the tree, as shown in \cref{scala_ast}.

\begin{figure}
  \consolefile{includes/reify.txt}
  \caption{Printing a Scala abstract syntax tree.}
  \label{scala_ast}
  \end{figure}

Back to our arithmetic expression trees, we can write a simple recursive
function evaluate trees to values as follows:

\begin{scalacode}
def eval(e: Expr): Int = e match {
  case Num(n) => n
  case Add(e1, e2) => eval(e1) + eval(e2)
  case Sub(e1, e2) => eval(e1) - eval(e2)
  case Mul(e1, e2) => eval(e1) * eval(e2)
  case Div(e1, e2) => eval(e1) / eval(e2)
}
\end{scalacode}

You could evaluate Scala programs by writing a recursive evaluation
function too. However, that's not what happens on your machines. To
understand how Scala is run, it will help to first examine how Java
programs are run.

\section{Stacks of Instructions}

\begin{figure}
\begin{minipage}{0.45\textwidth}
\begin{javacode}
class Simple {

  static int factorial(int n, int result) {
    if (n == 0) {
      return result;
    }
    else {
      return factorial(n - 1, result * n);
    }
  }

  public static void main(String[] args) {
    System.out.println("Hello, world!");
     System.out.println(factorial(5, 1));
  }

}
\end{javacode}
\caption{Java source code.}
\label{simplejava}
\end{minipage}
~\vrule~
\begin{minipage}{0.45\textwidth}
\begin{console}
class Simple {
  Simple();
    Code:
       0: aload_0
       1: invokespecial #1
       4: return

  static int factorial(int, int);
    Code:
       0: iload_0
       1: ifne          6
       4: iload_1
       5: ireturn
       6: iload_0
       7: iconst_1
       8: isub
       9: iload_1
      10: iload_0
      11: imul
      12: invokestatic  #2
      15: ireturn

  public static void main(java.lang.String[]);
    Code:
       0: getstatic     #3
       3: ldc           #4
       5: invokevirtual #5
       8: getstatic     #3
      11: iconst_5
      12: iconst_1
      13: invokestatic  #2
      16: invokevirtual #6
      19: return
}
\end{console}
\caption{Java bytecode.}
\label{javabytecode}
\end{minipage}
\caption{Disassembling Java.}
\end{figure}

Suppose you want to run the Java program in \cref{simplejava}. You
can't simply run it directly. Instead, you need to first
\emph{compile} it:
\begin{console}
javac Simple.java
\end{console}
which produces a file called \verb|Simple.class|. You can run this file by starting
the \emph{Java Virtual Machine} (JVM):
\begin{console}
java Simple
\end{console}
The latter command only uses \verb|Simple.class|. You can delete \verb|Simple.java| and
you'll still be able to run the program. Let's look into what the \verb|javac| and
\verb|java| commands do, before we return to Scala. Even if you've never used \verb|java| and \verb|javac| directly, and you've
only ever used Java within an IDE such as Eclipse or IDEA, you should know that these
IDEs run these commands under the hood.

The command \verb|javac| is a \emph{compiler} that translate Java source code to
\emph{byte-code}, which is stored in \verb|.class| files. Byte-code is not designed
to be human-readable, which is why \verb|.class| are in a binary format that will appear
are gibberish in a text editor. However, you can \emph{disassemble} the byte-code
and print it in a more human-readable format, by running \verb|javap -c Simple.class|.
This command produces the output shown in \cref{javabytecode}.

Observe that all variables in the original program
have been erased. Instead of using variables, Java byte-code uses a stack of values
and a list of instructions, where the instructions pop and push the values on the stack.

\begin{figure}
\begin{scalacode}
sealed trait Instruction
case class INum(n: Int) extends Instruction
case class IAdd() extends Instruction
case class IMul() extends Instruction
case class ISub() extends Instruction

def eval(code: List[Instruction], stack: List[Int]): List[Int] = (e, stack) match {
  case (INum(n), stack) => n :: stack
  case (IAdd(), n1 :: n2 :: stack) => n1 + n2 :: stack
  case (ISub(), n1 :: n2 :: stack) => n1 - n2 :: stack
  ...
}
\end{scalacode}
\caption{A stack-based evaluator for a list of arithmetic instructions.}
\end{figure}

Returning to arithmetic expressions, we can write a stack machine for arithmetic
instructions as shown in
\cref{stackmachine}. The command \verb|java Simple| runs
a (very sophisticated) stack machine that is based on the same basic principle illustrated
in this figure.

The following table shows some arithmetic expressions and equivalent 
instruction sequences

\begin{tabular}{l|l|l}
Concrete Syntax & Abstract Syntax & Instructions \\
\hline
\verb|1 + 2 + 3| & \lstinline[language=scala]|Add (Add (Num(1), Num(2)), Num(3))|
& \lstinline|List(INum(1), INum(2), IAdd(), INum(3), IAdd())| \\
\verb|1 + 2 * 3| & \lstinline[language=scala]|Add (Num(1), Mul (Num(2), Num(3)))|
& \lstinline|List(INum(1), INum(2), INum(3), IMul(), IAdd())| \\
\verb|(1 + 2) / 3| & \lstinline[language=scala]|Div (Add (Num(1), Num(2)), Num(3))|
& \lstinline|List(INum(1), INum(2), IAdd(), INum(3), IDiv())|
\end{tabular}

\begin{figure}
\begin{scalacode}
def compile(e: Expr): List[Instruction] = e match {
  case Num(n) => List(INum(n))
  case Add(e1, e2) => compile(e1) ++ compile(e2) ++ List(IAdd())
  case Sub(e1, e2) => compile(e1) ++ compile(e2) ++ List(ISub())
  case Mul(e1, e2) => compile(e1) ++ compile(e2) ++ List(IMul())
}
\end{scalacode}
\caption{A compiler from arithmetic expressions to arithmetic instructions.}
\label{arithcompiler}
\end{figure}

The command \verb|javac Simple.java| performs the kinds of translations shown
above automatically. The code in \cref{arithcompiler} compiles arithmetic expressions
to instructions using the same principles that \verb|javac| uses.

The Scala compiler translates Scala source code to byte-code for the Java Virtual Machine
in a similar way. The compiler is more complicated, because Scala is actually much more
complex than Java. However, once the \verb|.class| files have been generated, Scala
programs run in the same manner at Java programs.

Although we use \sbt{} in this course, it is also a simple frontend
for the Scala compiler (\verb|scalac|). You can run \verb|scalac| on
Scala source code to produce \verb|.class| files and then disassemble
or run these files in exactly the same way as Java code. In your
\sbt{} projects, there is a directory called
\verb|target/scala-2.11/classes| that has all the class files
generated from your program. \sbt{}, like any other IDE, puts
\verb|.class| files in a separate directory so that they don't clutter
your code.
