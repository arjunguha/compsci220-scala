\chapter{Lecture: Property-Based Testing}
\startlecture

\begin{instructor}

\section*{Lecture Outline}

\begin{enumerate}

\item  Sometimes, writing individual test cases isn't enough or isn't desirable.
It can help to think of properties instead of test cases.

\item Write some list processing functions: reverse, concat, map

\item Some properties of list processing functions: reverse and concat,
  concat distributes over map, reverse reverse is identity. \emph{Write these
    as functions}.

\item Show ScalaCheck. Show the generators generating things.

\item Write a tail recursive fibonnaci and check that the two implementations
are equivalent.

\item Write properties of join lists.

\end{enumerate}

\end{instructor}

\begin{figure}
  \scalafile{code/scalacheck/src/test/scala/ListTestSuite.scala}
  \caption{Checking properties of simple list processing functions.}
\end{figure}

\begin{figure}
  \scalafile{code/scalacheck/src/test/scala/FibTestSuite.scala}
  \caption{Checking that an optimized implementation of the fibonacci
    function is equivalent to the naive implementation.}
  
\end{figure}

\begin{figure}
  \scalafile{code/scalacheck/src/test/scala/JoinListSuite.scala}
  \caption{Checking properties of join lists.}
\end{figure}
