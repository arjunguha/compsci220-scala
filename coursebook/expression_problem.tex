\chapter{Object-Oriented vs. Functional Programming}
\startlecture

\begin{instructor}

\section*{Lecture Outline}

\begin{enumerate}

  

   \item Create a project called \lstinline|fshapes| and give it a
     name, organization, and version number \lstinline|build.sbt|.
     
   \item Define the \lstinline{fshapes} library, with
     \lstinline|Circle(radius)| and \lstinline|Square(side)| as case
     classes and a function \lstinline|inShape(s: Shape, x: Double, y: Double)| that
     uses pattern matching.

   \item Show \lstinline|publishLocal|.

   \item Create a project called \lstinline|jshapes| and fill \lstinline|build.sbt|
    as before.

    \item Show a Point class (not a case class). Show how to make fields
     public, show how to define methods and derived fields.

   \item Show that the entire body of the class can be thought of as a
     constructor, by putting a println in the body.

   \item Delete \lstinline|Point| and create two classes (not case-classes) that
     implement the \lstinline|Shape| trait.

   \item Translate the \lstinline|inShape| function into a method in \lstinline|Shape|
     that uses \lstinline|.isInstanceOf| and \lstinline|.asInstanceOf| instead of
     pattern matching. Say this is terrible.

   \item Rewrite \lstinline|inShape| using dynamic dispatch: note that
     the method bodies are exactly the same as the cases in the
     functional version! \textbf{Make the fields private.}

   \item Recipe: a function that pattern matches, turns into a method on the interface
     and each case becomes an implementation in the corresponding class.

   \item Exhaustivity check: Scala gives you an exhaustivity check in
     the functional code, but we don't have the exhaustivity check.

   \item Using libraries: suppose we found \lstinline|fshapes| and \lstinline|jshapes|
     on the Web. Which should we use? Is it just a matter of style, or are there objective
     criteria for preferring one over the other?

   \item A library is rarely ``complete''. We typically need to extend it. Suppose we
     don't want to edit the source code (or cannot do so).

   \item Add a new \emph{feature} to both libraries: the ability to
     double a shape in size. It is easy in the functional case. In the
     object-oriented case, we need to write a \lstinline|growShape|
     function using type-tests. \textbf{But, this fails because the
       fields are private.} We need to write a wrapper instead. Lots of extra code!

   \item Add anew \emph{data type} to both libraries: easy in the object-oriented
     case. It can't be done in the functional case without a wrapper. Lots of extra code!

   \item Which library should you pick? The one that supports the extensions that you
     anticipate needing to make. What if you want to do both? You're screwed!

\end{enumerate}

\end{instructor}

\section{Reading}

Programming in Scala, Chapter 6.1---6.11.\footnote{\url{http://www.artima.com/pins1ed/}}

\section{Object Oriented Scala}

\begin{figure}
\begin{minipage}{0.45\textwidth}
\begin{scalacode}
class Point(x: Double, y: Double) {

  def getX() = this.x

  def getY() = this.y

  def add(other: Point) = {
    new Point(this.x + other.getX(),
              this.y + other.getY())
  }

  def magnitude() = math.sqrt(x * x + y + y)
}
\end{scalacode}
\caption{A class for points.}\label{pointclass}
\end{minipage}
\quad\vrule\quad
\begin{minipage}{0.45\textwidth}
\begin{javacode}
class Point {
  private Double x;
  private Double y;

  public Point(Double x, Double y) {
    this.x = x;
    this.y = y;
  }

  public Double getX() {
    return this.x;
  }

  public Double getY() {
    return this.y;
  }

  public Point add(Point other) {
    return new Point(this.x + other.getX(),
                     this.y + other.getY());
  }

  public Double magnitude() = {
    return math.sqrt(x * x + y + y);
  }
}
\end{javacode}
\caption{A similar class in Java.}
\end{minipage}
\end{figure}

Although we've focused on functional programming so far, Scala supports
Java-style object-oriented programming too. For example, \cref{pointclass}
defines a \scalainline{Point} class (not a case class) with two private
fields\footnote{Recall that case class fields are public, which
is what allows us to use pattern matching. We cannot pattern match on ordinary classes,
even if their fields are public.} and four methods.

Notice that we didn't have to declare a constructor for \scalainline{Point}.
Scala creates a constructor automatically that initializes the values of
\scalainline{x} and \scalainline{y}.

We could make the fields public by writing:
\begin{scalacode}
class Point(val x: Double, val y: Double) { ... }
\end{scalacode}

We can make methods private by writing:
\begin{scalacode}
private def add(other: Point) = { ... }
\end{scalacode}

If we wanted to compute the value of a field, e.g., if we wanted the magnitude
to be a field whose value is calculated at construction, we could write:
%
\begin{scalacode}
class Point(x: Double, y: Double) {

  val magnitute = math.sqrt(x * x + y + y)

  ...
}
\end{scalacode}

In fact, you can think of the entire body of the class as the constructor.
For example, the following code prints each time a new point is constructed:

%
\begin{scalacode}
class Point(x: Double, y: Double) {

  println(s"Created a new Point with x = $x and y = $y.")

  ...
}
\end{scalacode}

These and other features of Scala's classes are described in depth in the
assigned reading. However, note that they are mostly syntactic differences.
The semantics of classes in Scala and Java are exactly the same (for now).

\section{Converting Functional Code to Object-Oriented Code}

\begin{figure}
\begin{scalacode}
sealed trait Shape
case class Circle(radius: Double) extends Shape // Center of the circle is (0, 0)
case class Square(side: Double) extends Shape // Bottom-left corner of the square is (0, 0)

object Shape {
  def inShape(s: Shape, x: Double, y: Double) = s match {
    case Circle(radius) => math.sqrt(x * x + y * y) <= radius
    case Square(side) => x >= 0 && x <= side && y >= 0 && y <= side
  }
}
\end{scalacode}
\caption{A Functional Shape Library}
\label{fshapes1}
\end{figure}

\Cref{fshapes1} shows a library for shapes, written in a functional style.
The library defines a sealed trait called \scalainline{Shape} and two
constructors for \scalainline{Circle} and \scalainline{Square} using
case classes. The library defines a function called \scalainline{inShape}
to test whether an $(x,y)$ coordinate is within the given shape. The function
uses pattern-matching to handle the two cases for circles and squares.

\begin{figure}
\begin{scalacode}
trait Shape {
  def inShape(x: Double, y: Double) = {
    if (this.isInstanceOf[Circle]) {
      math.sqrt(x * x + y * y) <= this.asInstanceOf[Circle].radius
    }
    else if (this.isInstanceOf[Square]) {
      val side = this.asInstanceOf[Square].side
      x >= 0 && x <= side && y >= 0 && y <= side
    }
    else {
      sys.error("unknown shape!")
    }
  }
}
class Circle(val radius: Double) extends Shape
class Square(val side: Double) extends Shape
\end{scalacode}
\caption{An object-oriented shape library: do not write code like this.}
\label{jshapesbad}
\end{figure}

\paragraph{Object-Oriented Refactoring Done Badly} To refactor this code into an
object-oriented style, we need to stop using case classes and pattern matching
and turn \scalainline{inShape} into a method instead of a function.
A literal translation of the code would be to turn the case classes
into classes with public fields, move the method into the \scalainline{Shape}
trait,\footnote{Think of this trait as an abstract class in Java. Traits are actually more powerful, but we'll get to that later.}, and rewrite the body to
use type-tests and casts.
 \Cref{jshapesbad} shows this refactoring of the
code.

\begin{notation}
In Scala:
\begin{scalacode}
this.instanceOf[Circle]
\end{scalacode}
is the same as
\javainline{this instanceof Circle} in Java.

In Scala:
\begin{scalacode}
this.asInstanceOf[Square]
\end{scalacode}
is the same as \javainline{(Square)this} in Java.
\end{notation}

You should already know that this is terrible code! It is an extremely bad idea
to use \javainline{instanceof} and type-casts in Java (and in Scala) because it
\emph{defeats the type system}. The point of the type system in Java (and Scala)
is to help you catch errors. When you write code like this, the type system
can't help you.
\textbf{You are not permitted to use \scalainline{isInstanceOf} and \scalainline{asInstanceOf},
since writing high-quality code is a major goal of this class.}

\paragraph{Object Oriented Refactoring Done Right}

\begin{figure}
\begin{scalacode}
trait Shape {
  def inShape(x: Double, y: Double): Boolean
}

class Circle(radius: Double) extends Shape {
   def inShape(x: Double, y: Double) = math.sqrt(x * x + y * y) <= radius
}

class Square(side: Double) extends Shape {
  def inShape(x: Double, y: Double) = x >= 0 && x <= side && y >= 0 && y <= side
}
\end{scalacode}
\caption{An object-oriented shape library done well.}
\label{jshapes1}
\end{figure}

The right way to build an object-oriented shape library is shown in \Cref{jshapes1}.
We've made
\scalainline{inShape} an abstract method in the trait
\scalainline{Shape}\footnote{Now, we're using the trait like an interface
in Java.} and each shape provides
an implementation. Note that we are not using any type-casts or type-tests and
that the shapes' fields are private, which is typically good practice in
object-oriented code. \textbf{Most significantly,} notice that the method
bodies are exactly the same as the cases in the function from \cref{fshapes1}.

Functional code that uses pattern matching can be converted to equivalent
object-oriented code by following this recipe:

\begin{enumerate}

  \item Functions that use pattern-matching become abstract methods on the type
  on which they pattern match.

  i.e., a function that looks like this:

  \begin{scalacode}
  def f(x: T, args ...): R = x match { ... }
  \end{scalacode}

  Becomes an abstract method:

  \begin{scalacode}
  trait T {
    def f(args ...): R
  }
  \end{scalacode}

  \begin{think}
  Notice that the \emph{function} $f$ takes an argument caled $x$, but
  the \emph{method} $f$ does not. Why not?
  \end{think}

  \item Each case in a function turns into an implementation of the abstract
  method in the corresponding class.

  i.e., a function that has a case like this:

  \begin{scalacode}
  def f(x: T, args ...): R = x match {
    case C(fields ...) => body
  }
  \end{scalacode}

  Becomes an implementation of the method $f$:

  \begin{scalacode}
  class C(fields ...) extends T {

    def f(args ...): R = body
  }
  \end{scalacode}

\end{enumerate}

\begin{think}
A key feature of pattern-matching is the \emph{exhuastivity check}. If
we forget to write a case in a function, Scala complains that pattern-matching
may not be exhaustive and actually lists all the cases that we have not
handled. Since we are no longer using pattern-matching, we no longer rely
on exhuastivity-checking. Is this okay? How do similar bugs manifest in
object-oriented code and can the Scala compiler catch them?
\end{think}

\section{Using Libraries}

Libraries and APIs are an important part of programming. Without libraries,
we'd be writing a lot of basic functionality from scratch. The Scala
and Java \emph{standard libraries} define a variety of useful data structures
and programming patterns and are available to all code that you write.
SBT makes it easy to use libraries from the Web.

For example,
\href{https://lihaoyi.github.io/upickle-pprint/upickle/}{uPickle} is a
handy library for converting Scala data structures to JSON, which is
often used in systems where programs in several languages need to communicate.
To use uPickle in a project, create a file called \texttt{build.sbt} in the
root directory of the project and add the line:
%
\begin{scalacode}
libraryDependencies += "com.lihaoyi" %% "upickle" % "0.3.8"
\end{scalacode}
%
After restarting SBT, you can use uPickle as documented on its website.
\href{http://mvnrepository.com}{Maven Central} has thousands of libraries
that you can use in a similar way.

\section{Writing Libraries}

Any SBT project can be packaged into a library by simply giving it
a name, organization, and version number in the \texttt{build.sbt} file.
For example, we could use the following \texttt{build.sbt} for our
functional shape library:

\begin{scalacode}
name := "fshapes"
organization := "compsci220"
version := "1.0"
\end{scalacode}

Before publishing a library on the Web, it is a good idea to test it by
publishing it locally. We can run \texttt{sbt publish-local} to do so.

When it completes, we can test the library by using it from another
project. To do so, we'd have to add this line to its \texttt{build.sbt}
file:
\begin{scalacode}
libraryDependencies += "compsci220" %% "fshapes" % "1.0"
\end{scalacode}

It takes a little more work to publish libararies online.
If you want to learn how to do so, we recommend using
\href{https://github.com/softprops/bintray-sbt}{Bintray SBT}. (It is a lot
simpler than using Maven Central directly.)

\section{Reusing and Extending Libraries}

Imagine that we wanted to use a library of shapes and we'd found both the functional
and object-oriented shape library we wrote earlier on the Web. We can use
either one by editing the \texttt{build.sbt} file:

\begin{scalacode}
libraryDependencies += "compsci220" %% "fshapes" % "1.0"
\end{scalacode}
or
\begin{scalacode}
libraryDependencies += "compsci220" %% "jshapes" % "1.0"
\end{scalacode}

So, which one should we use? Recall that both libraries implement the same
features: only two shapes and just one method/function to calculate if
a point is within a shape. So, perhaps the choice is just a matter of taste:
i.e., whether we prefer to think functionally or in an object-oriented way.

But, let's consider the choices more carefully. In particular, suppose neither
library was quite right and we needed to extend them in some way. In fact,
our libraries are so small, it's easy to imagine that we'd want a larger
variety of shapes and other functions or methods. So, let's see what it
takes to extend each library.

Remember that these are meant to be libraries that someone else wrote that
we are re-using in our own project. So, we do not want to read or modify their code. (Imagine that
the library has millions of lines of code). In fact, it may be a closed-source
library, so the original code may not even be available for us to read or modify.
So, imagine that all we have to work with is the Scaladoc and some example code.

We'll consider two kinds of extensions:

\begin{itemize}

  \item \emph{Extending the data model} by adding support for rectangles.

  \item \emph{Extending the feature set} by adding a routine to
  double the size of a shape.

\end{itemize}

\subsection{Extending the Functional Shape Library}

It is easy to write a new function to double the size of a shape. In our
own project, we could simply write:

\begin{scalacode}
object FunctionalExtension {
  def growShape(s: Shape): Shape = s match {
    case Circle(radius) => Circle(radius * 2)
    case Square(side) => Square(side * 2)
  }
}
\end{scalacode}

Now, let's add support for rectangles. The obvious thing to do is write:
\begin{scalacode}
import fshapes._
case class Rectangle(width: Double, height: Double) extends Shape
\end{scalacode}
But, this code does not type-check, because \scalainline{Shape} was \emph{sealed}
earlier. This may appear annoying now, but its a good thing that it doesn't work.
Remember that the \scalainline{inShape} function in the library doesn't
have a case for \scalainline{Rectangle}. How could it? Someone else wrote it
and didn't anticipate that we wanted support for rectangles. Therefore, by
preventing us from extending \scalainline{Shape}, Scala is ensuring that
the library code doesn't fail. (i.e., Scala is ensuring that the exhuastivity
analysis it did on the library remains valid.)

\begin{figure}
\begin{scalacode}
sealed trait ShapeExt
case class Rectangle(width: Double, height: Double) extends ShapeExt
case class OtherShape(s: Shape) extends ShapeExt

object FunctionalExtension {
  def growShape(s: ShapeExt): ShapeExt = s match {
    case OtherShape(Circle(radius)) => OtherShape(Circle(radius * 2))
    case OtherShape(Square(side)) => OtherShape(Square(side * 2))
    case Rectangle(width, height) => Rectangle(width * 2, height * 2)
  }

  def inShape(s: ShapeExt, x: Double, y: Double): Boolean = s match {
    case OtherShape(shape) => Shape.inShape(shape, x, y)
    case Rectangle(width, height) => x >= 0 && y >= 0 && x <= width && y <= height
  }
}
\end{scalacode}
\caption{Extending fshapes with rectangles.}
\label{fshapesext}
\end{figure}

However, all is not lost. We can add support for rectangles by \emph{wrapping}
the previously defined \scalainline{Shape} type in a new type. \Cref{fshapesext}
shows how to do this. The idea is quite straightforward. We simply
have to define a new \scalainline{inShape} function that handles rectangles
and invokes the original \scalainline{inShape} function on other shapes.

Now, imagine doing this on a large library with hundreds of functions. It would
be extremely tiresome. But, this illustrates a \emph{fundamental limitation of
functional programming}: it is easy to extends a library by adding new
functions, but it is hard to extend a library to support new kinds of data.

\begin{think}
Imagine that the original library had a \scalainline{growShape} function
and that we were trying to write the \scalainline{inShape} function.
Would wrapping still work?

Suppose the original library had a constructor for creating composite
shapes:
\begin{scalacode}
case class Overlay(top: Shape, bottom: Shape) extends Shape
\end{scalacode}
Would wrapping still work?
\end{think}

\subsection{Extending the Object-Oriented Shape Library}

It is easy to extend the object-oriented library with rectangles. All we
have to do is create a new class that extends \scalainline{Shape}
and define its \scalainline{inShape} method:
%
\begin{scalacode}
class Rectangle(width: Double, height: Double) extends Shape {
  def inShape(x: Double, y: Double): Boolean = x >= 0 && y >= 0 && x <= width && y <= height
}
\end{scalacode}

Now, let's try to add a feature to double the size of shapes. The object-oriented
way is to have a \scalainline{growShape} method in the \scalainline{Shape}
trait. But, we can't do that because \scalainline{Shape} was defined in
a library that someone else wrote (and we don't have the source code).

\begin{figure}
\begin{scalacode}
object JShapesExtension {
  def growShape(shape: Shape) {
    if (this.isInstanceOf[Circle]) {
      new Circle(this.asInstanceOf[Circle].radius * 2)
    }
    else if (this.isInstanceOf[Square]) {
      new Square(this.asInstanceOf[Square].side * 2)
    }
    else if (this.isInstanceOf[Rectangle]) {
      new Rectangle(this.asInstanceOf[Rectangle].width * 2, this.asInstanceOf[Rectangle].height * 2)
    }
    else {
      sys.error("unknown shape!")
    }
  }
}
\end{scalacode}
\caption{Using type-tests and type-casts to grow shows. Does not compile.}
\label{jshapesext1}
\end{figure}

Instead, we could write a \scalainline{growShape} function using
type-tests and type-casts, as shown in \cref{jshapesext1}. We've already
discussed why this is terrible code. Moreover, it has a simpler problem: it
does not compile.

\begin{think}
Why doesn't the code in \cref{jshapesext1} compile?
\end{think}

\begin{figure}
\begin{scalacode}
trait Growable {
  def growShape(): Shape
}

class MyCircle(radius: Double) extends Circle(radius) with Growable {
  def growShape(): MyCircle = new MyCircle(radius * 2)
}

class MySquare(side: Double) extends Square(side) with Growable {
  def growShape(): MySquare = new MySquare(side * 2)
}

class Rectangle(width: Double, height: Double) extends Shape with Growable {
  def inShape(x: Double, y: Double: Boolean = x >= 0 && y >= 0 && x <= width && y <= height

  def growShape(): Rectangle = new Rectangle(width * 2, height * 2)
}
\end{scalacode}
\caption{Using inheritance to extend the library.}
\label{jshapesext2}
\end{figure}

An alternative approach uses inheritance, as shown
in \cref{jshapesext2}. The key idea is to create  new types for circles
and squares that inherit from the existing libraries.
Again, imagine doing this at scale: if the library had dozens of objects
and dozens of methods, this would be extremely tedious.
 This illustrates a \emph{fundamental limitation of
object-oriented programming}: it is easy to extends a library with new
kinds of data (new classes), but it is hard to extend a library to support
new methods.

\begin{think}
Suppose the original library had a class that represented composite
shapes:
\begin{scalacode}
class Overlay(top: Shape, bottom: Shape) extends Shape { ... }
\end{scalacode}
Would wrapping work?
\end{think}

\section{Perspective}

These two examples illustrate that functional
and object-oriented styles involve fundamental tradeoffs when it comes
to extending and re-using code. There are certain kinds of extensions
that can be done in object-oriented style that cannot be done in
functional style, and vice versa.  It is misguided to argue that
one style is better than the other. What matters is how you expect your code
to be used (and extended). One could argue that a language that forces you to
pick one style of programming limits the kinds of code re-use that are possible.
