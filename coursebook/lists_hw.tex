\newhw{Introduction to Scala}

This assignment has several ``finger exercises'' that introduce you
to functional programming in Scala.

\section{Setup}

Before you start programming, you need to complete a few preliminary steps.

\begin{enumerate}

\item Follow the directions on the course website to install Scala and SBT.

\item Using the command-line, create a directory for your assignment (e.g., the
  \verb|hw1| directory). Within this directory, create the directories
  \texttt{project}, \verb|src/main/scala| and \verb|src/test/scala|. For example, you could use
  the following commands:
  %
  \begin{console}
  mkdir hw1
  cd hw1
  mkdir project
  mkdir src
  mkdir src/main
  mkdir src/main/scala
  mkdir src/test
  mkdir src/test/scala
  \end{console}

\item Using a text editor, create the file \texttt{project/plugins.sbt} with
the following contents:

\scalafile{../hw/lists/template/project/plugins.sbt}

   \item Using a text editor, create the file \verb|src/main/scala/Lists.scala|
   with the following contents:
   %
   \scalafile{../hw/lists/template/src/main/scala/Lists.scala}

   \item Using a text editor, create the file \verb|src/test/scala/TestSuite.scala|
   with the following contents:
   %
   \scalafile{../hw/lists/template/src/test/scala/TestSuite.scala}

  \item From the command-line, start \sbt{} and run the test suite. You should
  see output that looks like this:

  \lstinputlisting[language=console]{../hw/lists/template/test-output.txt}

  \noindent \emph{There should be no errors or warnings printed.}

\end{enumerate}

\section{Exercises}

For this assignment, you'll be writing several list-processing
functions. You must place these within the \verb|Lists| object that
you created above. You must write tests cases, within the \verb|TestSuite|
class that you created above.

\begin{enumerate}

  \item Write a function called \scalainline{sumDouble} that consumes a
  \scalainline{List[Int]} and produces an \scalainline{Int}. The produced
  value should be double the sum of the list of integers.

  \item Write a function called \scalainline{removeZeroes} that consumes a
  \scalainline{List[Int]} and produces a \scalainline{List[Int]}.
  The produced list should be the same as the input list, but with all zeroes
  removed. The function must not change the order of elements.

  \item Write a function called \scalainline{countEvens} that consumes a
  \scalainline{List[Int]} and produces an \scalainline{Int} that represents
  that number of even numbers in the input list.

  \item Write a function called \scalainline{removeAlternating} that consumes a
  \scalainline{List[String]} and produces a \scalainline{List[String]} that
  has every other element in the input list.

  The first element of the input list must be in the output list.
  For example:
  \begin{scalacode}
  assert(removeAlternating(List("A", "B")) == List("A"))
  assert(removeAlternating(List("A", "B")) != List("B"))
  \end{scalacode}

  The function must not change the order of elments.

  \item Write a function called \scalainline{isAscending} that consumes a
  \scalainline{List[Int]} and produces a \scalainline{Boolean} that is
  \scalainline{true} if the numbers in the input list are in ascending order.
  Note that the input may have repeated numbers.

  \item Write a function called \scalainline{addSub} that consumes a
  \scalainline{List[Int]} and produces an \scalainline{Int}. The
  function should add all the elements in even position and subtract all the
  elements in odd position.

  Note that the first element of a list is considered
  ``zeroth'' element, thus it is in even position.
  For example, \scalainline{addSub(List(10, 20, 30, 40))} should be
  \scalainline{10 - 20 + 30 - 40}.

  \item Write a function called \scalainline{alternate} that consumes \emph{two}
  \scalainline{List[Int]} arguments and produces a \scalainline{List[Int]}.
  The elements of the resulting list should alternate between the elements of
  the arguments. You may assume that the two arguments have the same
  length.

  For example:

  \begin{scalacode}
  assert(alternate(List(1, 3, 5), List(2, 4, 6)) == List(1, 2, 3, 4, 5, 6))
  \end{scalacode}

  \item Write a function called \scalainline{fromTo} that takes
  two \scalainline{Int}s as arguments and produces a \scalainline{List[Int]}.
  The value of \scalainline{fromTo(x, y)} should be the list of consecutive
  integers that start from and include $x$, going up to and excluding $y$.
  You may assume that $x < y$.

  For example:
  \begin{scalacode}
  assert(fromTo(9, 13) == List(9, 10, 11, 12))
  \end{scalacode}

  \item Write the following function:

  \begin{scalacode}
  def insertOrdered(n: Int, lst: List[Int]): List[Int]
  \end{scalacode}

  Assuming that \scalainline{lst} is in ascending order,
  \scalainline{insertOrdered} should produce a list that is the same as
  the input, but with $n$ inserted such that the order is preserved.
  For example, \scalainline{insertOrdered(5, List(1, 3, 7, 9))} should be
  \scalainline{List(1, 3, 5, 7, 9)}.

  You should assume that \scalainline{lst} is in ascending order. Your function
   may produce any result or even throw an exception if it is not.

   \item Write the following function:

   \begin{scalacode}
   def sort(lst: List[Int]): List[Int]
   \end{scalacode}

   The result should be the sorted input list.

\end{enumerate}

\section{Hand In}

From \sbt{}, run the command \texttt{submit}. The command will create
a file called \texttt{submission.tar.gz} in your assignment directory.
Submit this file using Moodle.

For example, if the command runs successfully, you will see output similar
to this:
%
\begin{console}
Created submission.tar.gz. Upload this file to Moodle.
[success] Total time: 0 s, completed Jan 17, 2016 12:55:55 PM
\end{console}

\textbf{Note:}  The command will not allow you to submit code that does not
compile. If your code doesn't compile, you will receive no credit for the
assignment.


