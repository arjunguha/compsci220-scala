\newhw{Regular Expressions}

In this assignment, you'll write several handy regular expressions. The assignment concludes with some more challenging problems that illustrate the kind of computations that are possible with regular expressions.

\section{Preliminaries}

You should create a directory-tree that looks like this:

\dirtree{%
.1 ./regexes.
.2 build.sbt.
.2 project.
.3 plugins.sbt.
.2 src.
.3 main.
.4 scala.\DTcomment{Your solution goes here}.
.3 test.
.4 scala\DTcomment{Yours tests go here}.
}

Your \texttt{build.sbt} file must have exactly these lines:

\scalafile{../hw/regex/template/build.sbt}

The \texttt{project/plugins.sbt} file must have exactly this line:

\scalafile{../hw/regex/template/project/plugins.sbt}

\section{Programming Tasks}

\begin{figure}
\scalafile{../hw/regex/template/src/main/scala/Solution.scala}
\caption{Template for the regular expressions.}
\label{regex_template}
\end{figure}

You may assume that your regular expressions will only be used to match complete strings. Therefore, you don't need to use the \texttt{\^} and \texttt{\$} metacharacters. A simple way to test if a regular expression \scalainline{regex} exactly matches a string \scalainline{str}, is to write \scalainline{regex.pattern.matcher(str).matches()}.

You should use the template in \cref{regex_template} for your solution and fill in the regular expressions as follows:

\begin{enumerate}

  \item Define the regular expression \scalainline{notAlphanumeric}, which only matches strings that don't contain any letter or digit.

  \item Define the regular expression \scalainline{time}, which only matches times written as five-character strings \texttt{HH:MM}, where the hours range from 00--23 and the minutes from 00--59.

  \item Define the regular expression \scalainline{phone}, which only matches phone numbers in the format \texttt{(XXX) XXX-XXXX}, where the letter \texttt{X} is a placeholder for a digit.

  \item Define the regular expression \scalainline{zip}, which matches either five-digit or nine-digit ZIP codes. i.e., strings in the format \texttt{XXXXX} or \texttt{XXXXX-XXXX}.

  \item Define the regular expression \scalainline{comment}, which only matches strings that start with \texttt{/*} and end with \texttt{*/}.

  \item Define the regular expression \scalainline{numberPhrase}, which only matches the strings \texttt{twenty}, \scalainline{twenty-one}, \scalainline{twenty-three}, $\cdots$, \scalainline{ninety-nine} (don't forget the hyphen).

  \item Define the regular expression \scalainline{roman}, which only matches strings that represent the numbers 0---39 in roman numerals. i.e., the string may only contain the characters \texttt{I}, \texttt{V}, and \texttt{X}, with the usual constraints on roman numerals.


  \item Define the regular expression \scalainline{date}, which only matches dates written as ten-character strings \texttt{YYYY-MM-DD}. For example, \texttt{2016-04-01} represents April 1 2016, whereas \texttt{2016-04-31} is not a valid date. You should account for leap years by assume that Feb 29th is valid if the year is evenly divisible by four. E.g., \texttt{2016-02-29} is a valid date but \texttt{2017-02-29} is invalid.

  \item The \emph{parity} of a number is the sum of the digits of the number. For example, the parity of $203$ is $2 + 0 + 3 = 5$ (odd) and the parity of $307$ is $3 + 0 + 7 = 10$ (even). Define the regular expression \scalainline{evenParity} that only matches numbers with even parity. You may find it convenient to accept numbers with leading zeroes.

\end{enumerate}

\section{Check Your Work}

\begin{figure}
\scalafile{../hw/regex/template/src/test/scala/TrivialTests.scala}
\caption{Your solution must pass this test suite with no modifications.}
\label{regex_trivial_tests}
\end{figure}

\Cref{regex_trivial_tests} is a trivial test suite that simply ensures that you've defined the right object.

\section{Hand In}

From \sbt{}, run the command \texttt{submit}. The command will create
a file called \texttt{submission.tar.gz} in your assignment directory.
Submit this file using Moodle.

For example, if the command runs successfully, you will see output similar
to this:
%
\begin{console}
Created submission.tar.gz. Upload this file to Moodle.
[success] Total time: 0 s, completed Jan 17, 2016 12:55:55 PM
\end{console}

\textbf{Note:}  The command will not allow you to submit code that does not
compile. If your code doesn't compile, you will receive no credit for the
assignment.

