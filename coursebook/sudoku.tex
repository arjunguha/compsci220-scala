\newhw{Sudoku}

For this assignment, you will write a Sudoku solver. To do so, you will
(1) use Scala collections extensively, (2) implement a \emph{backtracking
search} algorithm, and (3) implement \emph{constraint propagation}.


\section{Preliminaries}

We assume you know how to play Sudoku. If you don't, you should play a few
games by hand, before attempting this assignment.

The support code for this assignment is in the \scalainline{hw.sudoku}
package. You should create a directory-tree that looks like this:

\dirtree{%
.1 ./sudoku.
.2 project.
.3 plugins.sbt.
.2 src.
.3 main.
.4 scala\DTcomment{Your solution goes here}.
.3 test.
.4 scala\DTcomment{Yours tests go here}.
}

The \texttt{project/plugins.sbt} file must have exactly this line:

\scalafile{../hw/sudoku/template/project/plugins.sbt}

\section{Overview}

A Sudoku board is a 9-by-9 grid, where each cell is either blank or has a
value in the range 1--9. For this assignment, we'll use strings of length
81 to represent Sudoku boards, where each block of nine characters
represents a successive row. For example, the following string:

\begin{scalacode}
val puzzle =
  "....8.3...6..7..84.3.5..2.9...1.54.8.........4.27.6...3.1..7.4.72..4..6...4.1...3"
\end{scalacode}

Represents the following board:
\[
\small
\begin{array}{|c|c|c||c|c|c||c|c|c|}
\hline
  &   &   &   & 8 &   & 3 &    & \\
\hline
  & 6 &   &   & 7 &   &   & 8 & 4 \\
\hline
  & 3 &   & 5 &   &   & 2 &   & 9 \\
\hline \hline
  &   &   & 1 &   & 5 & 4 &   & 8 \\
\hline
  &   &   &   &   &   &   &   &   \\
\hline
4 &   & 2 & 7 &   & 6 &   &   &   \\
\hline \hline
3 &   & 1 &   &   & 7 &   & 4 &   \\
\hline
7 & 2 &   &   & 4 &   &   & 6 &   \\
\hline
  &   & 4 &   & 1 &   &   &   & 3 \\
\hline
\end{array}
\]

Here are some more examples:

\begin{scalacode}
".43.8.25.6.............1.949....4.7....6.8....1.2....382.5.............5.34.9.71."
"2...8.3...6..7..84.3.5..2.9...1.54.8.........4.27.6...3.1..7.4.72..4..6...4.1...3"
"..3.2.6..9..3.5..1..18.64....81.29..7.......8..67.82....26.95..8..2.3..9..5.1.3.."
"1..92....524.1...........7..5...81.2.........4.27...9..6...........3.945....71..6"
\end{scalacode}

Your first task is to parse strings that represent Sudoku boards. You may
assume that all strings represent solvable boards and that the string has
exactly 81 characters. This part of the assignment should be trivial.

Solving Sudoku puzzles is much harder, but we'll walk you through it.

\emph{Backtracking search} is a recursive algorithm that operates as
follows. Given a a Sudoku board, $B$:

\begin{itemize}
\item If $B$ is already filled completely and a solution, return the solution.
\item If $B$ is an invalid board (e.g., two 2s in a row), abort.
\item Otherwise, generate a list of all boards that fill in one more square of
  $B$. For each generated board, recursively apply the search function:
  \begin{itemize}
  \item If any board produces a valid solution, return that board
  \item If no board produces a solution, abort and return
  \end{itemize}
\end{itemize}


This approach will work in principle. But, in practice there are too many
boards to search: there are 81 squares, and each square can hold 10 values
(a digit or blank). Therefore, there are $10^{81}$ possible boards, which
exceeds the
\href{http://en.wikipedia.org/wiki/Observable_universe#Matter_content_.E2.80.94_number_of_atoms}{number of atoms in the observable universe}.

To actually solve Sudoku problems, we need to combine backtracking search
with \emph{constraint propogation}. When you play Sudoku yourself, every
time you fill a digit into a cell, you can eliminate that digit from
several other cells. For example, if you fill 2 into the top-left corner
of the board above, you can eliminate 2 from the first row, first column,
and first box. i.e., there is no point even trying to place 2 in those
spots, since the 2 in the corner \emph{constrains} those cells.


We'll augment backtracking search with constraint propagation, which
implements this intuition. The key idea is to store not the value at a
cell, but the \emph{set of values} that may be placed in a cell. For
example, on the empty board the values 1---9 may be placed at any cell:

\[
\small
\begin{array}{|c|c|c||c|c|c||c|c|c|}
\hline
123456789 & 123456789 & 123456789 & 123456789 & 123456789 & 123456789 & 123456789 & 123456789 & 123456789 \\
\hline
123456789 & 123456789 & 123456789 & 123456789 & 123456789 & 123456789 & 123456789 & 123456789 & 123456789 \\
\hline
123456789 & 123456789 & 123456789 & 123456789 & 123456789 & 123456789 & 123456789 & 123456789 & 123456789 \\
\hline \hline
123456789 & 123456789 & 123456789 & 123456789 & 123456789 & 123456789 & 123456789 & 123456789 & 123456789 \\
\hline
123456789 & 123456789 & 123456789 & 123456789 & 123456789 & 123456789 & 123456789 & 123456789 & 123456789 \\
\hline
123456789 & 123456789 & 123456789 & 123456789 & 123456789 & 123456789 & 123456789 & 123456789 & 123456789 \\
\hline \hline
123456789 & 123456789 & 123456789 & 123456789 & 123456789 & 123456789 & 123456789 & 123456789 & 123456789 \\
\hline
123456789 & 123456789 & 123456789 & 123456789 & 123456789 & 123456789 & 123456789 & 123456789 & 123456789 \\
\hline
123456789 & 123456789 & 123456789 & 123456789 & 123456789 & 123456789 & 123456789 & 123456789 & 123456789 \\
\hline
\end{array}
\]

With this representation, when we place a value at a cell, we eliminate it
from the other cells in the same row, column, and box (collectively known
as the \emph{peers} of a cell). For example, if we place 5 at the top-
left corner of the empty board, we can eliminate 5 from the peers of the
top-left corner:


\[
\small
\begin{array}{|c|c|c||c|c|c||c|c|c|}
\hline
5 & 1234~6789 & 1234~6789 & 1234~6789 & 1234~6789 & 1234~6789 & 1234~6789 & 1234~6789 & 1234~6789 \\
\hline
1234~6789 & 1234~6789 & 1234~6789 & 123456789 & 123456789 & 123456789 & 123456789 & 123456789 & 123456789 \\
\hline
1234~6789 & 1234~6789 & 1234~6789 & 123456789 & 123456789 & 123456789 & 123456789 & 123456789 & 123456789 \\
\hline \hline
1234~6789 & 123456789 & 123456789 & 123456789 & 123456789 & 123456789 & 123456789 & 123456789 & 123456789 \\
\hline
1234~6789 & 123456789 & 123456789 & 123456789 & 123456789 & 123456789 & 123456789 & 123456789 & 123456789 \\
\hline
1234~6789 & 123456789 & 123456789 & 123456789 & 123456789 & 123456789 & 123456789 & 123456789 & 123456789 \\
\hline \hline
1234~6789 & 123456789 & 123456789 & 123456789 & 123456789 & 123456789 & 123456789 & 123456789 & 123456789 \\
\hline
1234~6789 & 123456789 & 123456789 & 123456789 & 123456789 & 123456789 & 123456789 & 123456789 & 123456789 \\
\hline
1234~6789 & 123456789 & 123456789 & 123456789 & 123456789 & 123456789 & 123456789 & 123456789 & 123456789 \\
\hline
\end{array}
\]

This procedure will significantly reduce the number of boards that need to
be visited.

\section{Programming Task}

\begin{figure}
\scalafile{../hw/sudoku/template/src/main/scala/Solution.scala}

\caption{Template for Sudoku}\label{sudokutemplate}
\end{figure}


Your task is to implement the \scalainline{SolutionLike} and
\scalainline{BoardLike} traits. You can use the template in
\cref{sudokutemplate} to do so.

We recommend proceeding in this order and testing as you go along:

\begin{enumerate}

\item Implement \scalainline{Solution.peers}. \scalainline{peers(r, c)}  produces the coordinates of all cells in the same row as \scalainline{r}, same column as \scalainline{c}, and same block as \scalainline{(r,c)}.

   Do not include \scalainline{(r, c)} in the set of its own peers. i.e., the value of
   the expression
   \scalainline{peers(r,c).contains((r, c))} should be \scalainline{false}.

\item Implement \scalainline{Solution.parse} You  should assume that the input string has exactly 81 characters, where each character is a digit or a period. Each successively block of nine characters represents a row (i.e., row-major order).

   As the template suggests, you need to store the set of available values at
   each cell instead of the value at the cell. You'll need to define
   the empty board as the map with all values available at each cell and
   implement \scalainline{parse} using \scalainline{peers} as a helper function.

\item Implement \scalainline{Board.valueAt}. You should produce the digit stored at the given row and column or \scalainline{None} if it is blank.
\item Implement \scalainline{Board.isSolved} and \scalainline{Board.isUnsolvable}. You may assume
   that the constraints represented by \scalainline{available} are valid. Therefore, a board
   is solved if every cell is constrained to exactly one value. Similarly,
   a board is unsolvable if any cell is constrained to the empty set of values
   (i.e., nothing can be placed in that cell).

\item Implement \scalainline{Board.place}. The application \scalainline{place(row, col, value)} produces a new board with
   \scalainline{value} placed at \scalainline{(row, col)} of \scalainline{this} board.  You may assume that
   the predicate \scalainline{availableValuesAt(row, col).contains(value)} is true.

   For the new board to be valid, you'll have to:

   \begin{enumerate}
   \item Remove \scalainline{value} from set of available values of each peer
      (i.e., \scalainline{peers(row, col)}).
   \item While doing (1), if a peer is constrained to exactly 1 value, remove
      that value from its peers.
   \end{enumerate}

\item Implement \scalainline{nextStates}, which returns the list of all boards that have
   exactly one additional value placed on the board.

   You should sort the returned list: ensure that boards with fewer available
   values occur earlier in the list.

\item Implement \scalainline{solve}. If \scalainline{this.isSolved} is true, then return \scalainline{Some(this)}.
   If not, iterate through the list of \scalainline{nextStates}, applying \scalainline{solve} to
   board. Return the first solution that you find. If no solution is found,
   return \scalainline{None}.

\end{enumerate}


\section{Solvable Boards}

Your solver will not be able to solve arbitrary Sudoku boards. But,
here are some boards that should work:

\begin{scalacode}
val fromCS121_1 = 
  "85....4.1......67...21....3..85....7...982...3....15..5....43...37......2.9....58"
val fromCS121_2 = 
  ".1.....2..3..9..1656..7...33.7..8..........89....6......6.254..9.5..1..7..3.....2"
val puz1 = 
  ".43.8.25.6.............1.949....4.7....6.8....1.2....382.5.............5.34.9.71."
val puz2 = 
  "2...8.3...6..7..84.3.5..2.9...1.54.8.........4.27.6...3.1..7.4.72..4..6...4.1...3"
\end{scalacode}

There are several sources of Sudoku puzzles on the Web. There are
50 purportedly easy puzzles on Peter Norvig's webpage:

\url{http://norvig.com/easy50.txt}

You can use this terminal command to translate them into the format for this assignment:

\consolefile{includes/curl-norvig}

\textbf{In addition, we encourage you to share solvable puzzles with each other on Piazza.}

\section{Check Your Work}

Here is a trivial test suite that simply checks to see if you've defined
the \scalainline{Solution} object with the right type:

\scalafile{../hw/sudoku/template/src/test/scala/TrivialTestSuite.scala}

You should place this test suite in \texttt{src/test/scala/TrivialTestSuite.scala}. This test suite must run as-is, or you may receive no credit for your solution.

\section{Hand In}

From \sbt{}, run the command \texttt{submit}. The command will create
a file called \texttt{submission.tar.gz} in your assignment directory.
Submit this file using Moodle.

For example, if the command runs successfully, you will see output similar
to this:
%
\begin{console}
Created submission.tar.gz. Upload this file to Moodle.
[success] Total time: 0 s, completed Jan 17, 2016 12:55:55 PM
\end{console}

\textbf{Note:}  The command will not allow you to submit code that does not
compile. If your code doesn't compile, you will receive no credit for the
assignment.


