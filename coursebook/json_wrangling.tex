\newhw{Data Wrangling with Json}

\noindent As a software engineer, you often have to work with large datasets.
Such data is used in everything from large machine learning applications, to
simple conversation bots. For this assignment you will be working
with the JSON format. JSON (or JavaScript Object Notation)\footnote{While
closely related to the JavaScript programming language, all major languages
support working with JSON through external libraries} is a widely used format
that encodes data in a readable textual format. You'll be working with
\href{https://www.yelp.com/dataset}{Yelp's academic dataset} to answer vital
questions such as ``What is the most popular restaurant in California?''.

\section{Preliminaries}
You should create a directory-tree that looks like this:
\dirtree{%
.1 ./wrangling.
.2 build.sbt.
.2 project.
.3 plugins.sbt.
.2 src.
.3 main.
.4 scala.\DTcomment{Your solution goes here}.
.3 test.
.4 scala\DTcomment{Yours tests go here}.
}
The \texttt{project/plugins.sbt} file must have exactly this line:
\begin{scalacode}
addSbtPlugin("edu.umass.cs" % "compsci220" % "1.0.0")
\end{scalacode}

You'll also need to download a shortened yelp dataset from the assignment page
on moodle.

\section{The Data}
For this assignment, you'll we working with Json data encoded using Scala case
classes. Json data can contain the following parts:
\begin{itemize}
    \item \textit{Constants}: Numbers, Strings (double quoted), Booleans, and
    \texttt{null} can be directly represented in Json. Example: ``cat'', 1.0,
    \texttt{true}.

    \item \textit{null}: \texttt{null} is a special value in Json that
    represents a missing field.

    \item \textit{Arrays}: A Json array is an array that itself contains other
    Json values. Json arrays are not type-constrained, that is, an array can
    contain multiple types of Json values. Example: [1.0, 2, null]

    \item \textit{Objects}: A Json object (or a Json dictionary) contains
    \textit{pairs} of keys and values separated by commas. Each pair is
    represented by a double quoted string followed by a colon followed by any
    Json value. Example:
    \begin{scalacode}
    {
        "array": [1,2,3],
        "animal": "cat",
        "range": { "start": 1, "end": 10 }
    }
    \end{scalacode}
\end{itemize}

\noindent The Scala case classed that encode these Json values can be found
in the homework package documentation.
\\ \\
In the dataset file, each line has a Json object. Each Json object looks like
this:
    \begin{scalacode}
    {
        "name": "Jade Palace",
        "city": "San Francisco",
        "state": "CA",
        "stars": 4.5,
        "review_count": 1140,
        "attributes": { ... },
        "categories": [ "Chinese", "TakeOut" ],
    }
    \end{scalacode}
The \texttt{build.sbt} file for this assignment is configured to load a library
for reading Json files. For example, the following program reads in the dataset
file and returns a list of \texttt{Json} objects:
\begin{scalacode}
import hw.json
val data: List[json.Json] = json.fromFile("yelp.json")
\end{scalacode}

\section{Programming Task}

Your programming task is to implement the functions described below. You should
place all functions within an object called \texttt{Wrangling}.

\begin{enumerate}
    \item Write a function that consumes a list of Json objects and returns
    those records from a given state (state codes are two letters).
    \begin{scalacode}
    def fromState(data: List[Json], state: String): List[Json]
    \end{scalacode}

    \item Write a function that consumes a list of Json objects and returns
    those records that have rating less than or equal to the given bound:
    \begin{scalacode}
    def ratingLT(data: List[Json], rating: Int): List[Json]
    \end{scalacode}

    \item Write a function that consumes a list of Json objects and returns
    those records that have rating more than or equal to the given bound:
    \begin{scalacode}
    def ratingGT(data: List[Json], rating: Int): List[Json]
    \end{scalacode}

    \item Write a function that consumes a list of Json objects and returns
    those records that have the given category:
    \begin{scalacode}
    def category(data: List[Json], category: String): List[Json]
    \end{scalacode}

    \item Write a function that consumes a list of Json objects and returns
    a list of Json objects grouped by their home state:
    \begin{scalacode}
    def groupByState(data: List[Json]): Map[String, List[Json]]
    \end{scalacode}

    \item Write a function that consumes a list of Json objects and returns
    a list object grouped by their category. Note that if a restaurant has
    multiple categories, it should occur in each pair.
    \begin{scalacode}
    def groupByCategory(data: List[Json]): Map[String, List[Json]]
    \end{scalacode}

    \item Write a function that consumes a list of Json objects and returns
    the ``best'' establishment. The ``best'' establishment is the one that
    has the highest score and the most number of reviews. Note that the
    highest score in a list might not be 5.
    \begin{scalacode}
    def bestPlace(data: List[Json]): Json
    \end{scalacode}

    \item Each Json object contains an \texttt{attributes} field that
    \textit{might} contain itself contain an \texttt{Ambiance} field which
    itself is an object:
    \begin{scalacode}
    {   ...
        "attributes": {
            ...
            "Ambience": {
                "hipster": false,
                "trendy": false,
                "upscale": false,
                "casual": false
            }
        }
    }
    \end{scalacode}
    Write a function that consumes a list of Json objects and returns a list
    of Json objects with a given ambience. If an object doesn't have an
    \texttt{Ambience} field, exclude it from the list.
    \begin{scalacode}
    def hasAmbience(data: List[Json], ambience: String): List[Json]
    \end{scalacode}

\end{enumerate}

\noindent You'll notice that many of the functions produce a list of Json
objects. The key idea is that you can compose these functions together to
pose complex queries. For example, suppose \texttt{yelpJson} holds the
complete yelp data.

\begin{itemize}
\item The following query finds the best takeout restaurant in California:
    \begin{scalacode}
    bestPlace(category(fromState(yelpJson, "CA"), "takeout"))
    \end{scalacode}

\item The following query finds the best establishment in each state:
    \begin{scalacode}
    groupByState(yelpJson).map((state, els) => (s, best(els)))
    \end{scalacode}
\end{itemize}

\section{Hand In}

From \sbt{}, run the command \texttt{submit}. The command will create
a file called \texttt{submission.tar.gz} in your assignment directory.
Submit this file using Moodle.

For example, if the command runs successfully, you will see output similar
to this:
%
\begin{console}
Created submission.tar.gz. Upload this file to Moodle.
[success] Total time: 0 s, completed Jan 17, 2016 12:55:55 PM
\end{console}

\textbf{Note:}  The command will not allow you to submit code that does not
compile. If your code doesn't compile, you will receive no credit for the
assignment.


