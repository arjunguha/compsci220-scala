\chapter{Project: Data Wrangling with Json}
\savepdf{project-json-data-wrangling}

\noindent As a software engineer, you often have to work with large datasets.
Such data is used in everything from large machine learning applications, to
simple conversation bots. For this assignment you will be working
with the JSON format. JSON (or JavaScript Object Notation)\footnote{While
closely related to the JavaScript programming language, all major languages
support working with JSON through external libraries} is a widely used format
that encodes data in a readable textual format. You'll be working with
\href{https://www.yelp.com/dataset}{Yelp's academic dataset} to answer vital
questions such as ``What is the most popular restaurant in California?''.
\textbf{Note:} You are allowed to use Scala's builtin functions such as
\verb|List.map| and \verb|List.groupBy|.

\section{The Data}
For this assignment, you'll we working with Json data encoded using Scala case
classes. Json data can contain the following parts:
\begin{itemize}
    \item \textit{Constants}: Numbers, Strings (double quoted), and Booleans
    can be directly represented in Json. Example: ``cat'', 1.0, \texttt{true}.

    \item \textit{null}: \texttt{null} is a special value in Json that
    denotes a missing value.

    \item \textit{Arrays}: A Json array is an array that itself contains other
    Json values. Json arrays are not type-constrained, that is, an array can
    contain multiple types of Json values. Example: [1.0, 2, null]

    \item \textit{Objects}: A Json object (or a Json dictionary) contains
    \textit{pairs} of keys and values separated by commas. Each pair is
    represented by a double quoted string followed by a colon followed by any
    Json value. Example:
    \begin{scalacode}
    {
        "array": [1,2,3],
        "animal": "cat",
        "range": { "start": 1, "end": 10 }
    }
    \end{scalacode}
\end{itemize}

The dataset file you'll be working with has one Json object on each line. Each
Json object looks like this:
    \begin{scalacode}
    {
        "name": "Jade Palace",
        "city": "San Francisco",
        "state": "CA",
        "stars": 4.5,
        "review_count": 1140,
        "attributes": { ... },
        "categories": [ "Chinese", "TakeOut" ],
    }
    \end{scalacode}
The course plugin provides the \lstinline|hw.json| package to work with Json
values. The package defines several case classes that encode Json values. It
also defines helper functions to read and print Json values. For example,
the following code reads in the file \texttt{yelp.json} from the root of the
project directory:
\texttt{Json} objects:
\begin{scalacode}
import hw.json
val data: List[json.Json] = JsonHelper.fromFile("yelp.json")
\end{scalacode}

\noindent All the case class and helper functions are documented in the `Course
package documentation' link on moodle.

\section{Preliminaries}

Using the command-line, run the command
\verb|sbt new umass-compsci220/scala-json-data-wrangling.g8|. When prompted for
a project name, press ``Enter'' to accept the default name. This will create
the directory \verb|scala-json-data-wrangling|.
\\
\noindent You will also need to download \verb|yelp.json| provided with the
assignment and place it in the project directory. You can do this either by
manually moving the file after the download, or running the command:
\begin{verbatim}
mv ~/Downloads/yelp.json ~/scala-json-data-wrangling
\end{verbatim}

\noindent Within the file \verb|scala-json-data-wrangling/src/main/scala/Main.scala|,
implement the functions described below.

\section{Programming Task}

Before starting, make note of two things:

\begin{itemize}
\item Since a lot of the functions given make use of strings, it is
\textit{very easy to make a mistake}. Therefore, it is critical that you test
these functions before submission.

\item All functions take a \texttt{Json} value, not a \texttt{JsonDict} value.
Make sure you handle the case where the input value in a function is not
a \texttt{JsonDict} in every function. If the function returns a list of
Json values, simply exclude such a Json value from the list.
\end{itemize}

Your programming task is to implement the functions described below. You should
place all functions within an object called \texttt{Wrangling}.

\begin{enumerate}
    \item Write a function that extracts a value associated to the given key
    from a Json value. If the given json value is not a JsonDict, or if the
    JsonDict does not have the given key, return \verb|None|, otherwise return
    the value wrapped in \verb|Some|.
    \begin{scalacode}
    def key(json: Json, key: String): Option[Json]
    \end{scalacode}
    \item \textbf{(Covered in discussion.)} Write a function that consumes a
    list of Json objects and returns those records from a given state (state
    codes are two capital letters, "CA", "MA". Take a look at yelp.json for
    more examples).
    \begin{scalacode}
    def fromState(data: List[Json], state: String): List[Json]
    \end{scalacode}
    \item Write a function that consumes a list of Json objects and returns
    those records that have rating less than or equal to the given bound:
    \begin{scalacode}
    def ratingLT(data: List[Json], rating: Double): List[Json]
    \end{scalacode}
    \item Write a function that consumes a list of Json objects and returns
    those records that have rating more than or equal to the given bound:
    \begin{scalacode}
    def ratingGT(data: List[Json], rating: Double): List[Json]
    \end{scalacode}
    \item Write a function that consumes a list of Json objects and returns
    those records that have the given category:
    \begin{scalacode}
    def category(data: List[Json], category: String): List[Json]
    \end{scalacode}
    \item \textbf{(Covered in discussion.)} Write a function that consumes a
    list of Json objects and returns a Map from state to a list of Json values
    from that state.
    \begin{scalacode}
    def groupByState(data: List[Json]): Map[String, List[Json]]
    \end{scalacode}
    \item Write a function that consumes a list of Json objects and returns a
    map from category to a list of Json values from that category. Note that if
    a restaurant has multiple categories, it should occur in each pair.
    \begin{scalacode}
    def groupByCategory(data: List[Json]): Map[String, List[Json]]
    \end{scalacode}
    \item Write a function that consumes a list of Json objects and returns the
    ``best'' establishment. To get the ``best'' establishment, first get a list
    of Json objects with the highest ``stars'', then select the establishment
    with the highest ``review\_count'' from that list.
    \begin{scalacode}
    def bestPlace(data: List[Json]): Option[Json]
    \end{scalacode}
    \item Each Json object contains an \texttt{attributes} key that
    \textit{might} itself contain an \texttt{Ambience} (notice case) key which
    itself is an object:
    \begin{scalacode}
    {   ...
        "attributes": {
            ...
            "Ambience": {
                "hipster": false,
                "trendy": false,
                "upscale": false,
                "casual": false
            }
        }
    }
    \end{scalacode}
    Write a function that consumes a list of Json objects and returns a list
    of Json objects with a given ambience. If an object doesn't have an
    \texttt{Ambience} key, exclude it from the list.
    \begin{scalacode}
    def hasAmbience(data: List[Json], ambience: String): List[Json]
    \end{scalacode}
    \item \textbf{(Not graded.)} For the Json object we currently have, there
    is no way to uniquely identifying an object. Write a function that adds a
    new key called "uid" to every Json object. The value associated to the
    key should be unique to each object. \textbf{Do not} use variables,
    mutation, or the Scala's builtin random function to implement this. Do not
    modify the given list, instead produce a new list.
    \begin{scalacode}
    def addUid(json: List[Json]): List[Json]
    \end{scalacode}

\end{enumerate}

\noindent You'll notice that many of the functions produce a list of Json
objects. The key idea is that you can compose these functions together to
pose complex queries. For example, suppose \texttt{yelpJson} holds the
complete yelp data.

\begin{itemize}
\item The following query finds the best takeout restaurant in California:
    \begin{scalacode}
    bestPlace(category(fromState(yelpJson, "CA"), "takeout"))
    \end{scalacode}

\item The following query finds the best establishment in each state:
    \begin{scalacode}
    groupByState(yelpJson).map((state, els) => (s, best(els)))
    \end{scalacode}
\end{itemize}

\section{Hand In}

From \sbt{}, run the command \texttt{submit}. The command will create
a file called \texttt{submission.tar.gz} in your assignment directory.
Submit this file using Moodle.

For example, if the command runs successfully, you will see output similar
to this:
%
\begin{console}
Created submission.tar.gz. Upload this file to Moodle.
[success] Total time: 0 s, completed Jan 17, 2016 12:55:55 PM
\end{console}

\textbf{Note:}  The command will not allow you to submit code that does not
compile. If your code doesn't compile, you will receive no credit for the
assignment.


