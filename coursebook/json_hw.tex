\newhw{JSON Parser and Printer}

\section{Introduction}
In previous assignments, you've worked extensively with data in JSON format and
have manipulated it using Scala's case classes. However, as you might have
noticed, JSON is a text based encoding that needs to be \textit{parsed}. In
this assignment you'll be writing a parser that takes in valid JSON text and
converts it into Scala case classes. Since we will be using the JSON case
classes from previous assignments, you will be able to plug in your parser and
use it in place of ours!
\\ \\
To refresh your memory, here are some examples of the \textit{concrete syntax}
of JSON:
\begin{itemize}
    \item \lstinline|{"dogs": 1, "cats": 2}|
    \item \lstinline|{"cats": { "name": "Hilbert", "age": 4 }}|
    \item \lstinline|{"numbers": [1, 2, 3, 4]}|
    \item \lstinline|{"blue": true, "yellow": false, "red": null}|
\end{itemize}
The assignment is divided into two parts:
\begin{enumerate}
    \item \textbf{Parsing JSON}: For this part, we will make use of Scala's
    parser combinators to implement a parser that reads in a JSON string and
    converts it into our case classes.
    \item \textbf{Printing JSON}: For this part, we will convert our case classes
    back into a string. Note that since we make use of scala's case classes,
    we are \textit{guaranteed} to only output valid JSON.
\end{enumerate}

\section{Programming Task}
We will be using the grammar specified below. Note that \texttt{true, false,
null} are literal constants, i.e. they appear as exact identifier strings.
\begin{verbatim}
string := "[^"]*"

number := -?[0-9]+(.[0-9]+)?

pair := string : json

members :=
   | pair
   | pair , members

object :=
   | {}
   | { members }

elements :=
   | json
   | json , elements

array :=
   | []
   | [ elements ]

json :=
   | string
   | number
   | object
   | array
   | true
   | false
   | null
\end{verbatim}
To complete the assignment, You will need to fill out the following solution
template:
\scalafile{../hw/json_parsing/template/src/main/scala/Solution.scala}
We suggest proceeding in the following order:
\begin{enumerate}
    \item Implement \lstinline|JSONParser| by translating the grammar provided
    above to Scala's parser combinators.
    \item Implement \lstinline|JSONPrinter|.
\end{enumerate}
